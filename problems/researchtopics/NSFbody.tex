\documentclass[11pt]{article}
\input{header}
\newcommand{\plucker}{Pl\"{u}cker\ }
\newcommand{\tang}{tangential curve\ }
\newcommand{\tangs}{tangential curves\ }
\newcommand{\Tang}{Tangential curve\ }
\newcommand{\atang}{tangential $a$-curve\ }
\newcommand{\btang}{tangential $b$-curve\ }
\newcommand{\ctang}{tangential $c$-curve\ }
\newcommand{\acurve}{$a$-curve\ }
\newcommand{\bcurve}{$b$-curve\ }
\newcommand{\ccurve}{$c$-curve\ }
\newcommand{\atangs}{tangential $a$-curves\ }
\newcommand{\btangs}{tangential $b$-curves\ }
\newcommand{\ctangs}{tangential $c$-curves\ }
\SingleSpace

\setlength{\oddsidemargin}{0pt}
\setlength{\topmargin}{-.2in}	% should be 0pt for 1in
% \setlength{\headsep}{.5in}
\setlength{\textheight}{8.5in}
\setlength{\textwidth}{6.5in}
\setlength{\columnsep}{5mm}	% width of gutter between columns
\markright{NSF proposal: \today \hfill}
\pagestyle{myheadings}
% -----------------------------------------------------------------------------

\begin{document}

\title{Bitangents and the Geometric Modeling of Visual Events\\
	for Visibility Analysis, Lighting, and Motion}
\author{J.K. Johnstone}
\maketitle
	
\begin{abstract}
The bitangents of a scene of objects define many important loci:
including the visual events where the set of visible objects undergoes a qualitative change,
the paths of a shortest motion, and the boundaries of umbra, penumbra and complete light
in a lighting simulation.
Computation of the bitangents of curves and surfaces, and associated loci
such as bitangent developables of surfaces, has not received
much attention in the graphics or modeling literature.
Most work on visibility, lighting, and robot motion
has been restricted to polyhedra, where the bitangent problem is trivial.
The maturation of these areas to smooth objects demands a full understanding
of bitangents, and efficient algorithms for their computation.
\end{abstract}

\section{Introduction}

Consider a scene of smooth objects and a camera (or eye or robot)
moving about through this scene.
Certain objects will be visible to the camera while others, completely blocked by intervening
objects, will be invisible.
(Here we consider an object to be visible if any part of it is visible.)
There is a coherence to the visibility of the scene in the following sense.
From almost all viewpoints,
the qualitative visibility of the scene (the set of objects visible from
the viewpoint) is invariant to small changes of the viewpoint.
However, from some viewpoints, any positional change in the viewpoint
will qualitatively change the visibility, by making visible an object previously
invisible or making invisible a previously visible object.
These special viewpoints are called visual events.
For example, ---.
The visibility analysis of a scene is dependent on the computation
of these visual events.

Visual events are viewpoints at which the visibility of a scene
undergoes a qualitative change.

===

Tangent $\rightarrow$ Bitangent.
Silhouette $\rightarrow$ Bisilhouette.

===

The bitangent lines of a 2-dimensional or 3-dimensional curved environment
are fundamental to visibility, shortest paths, and lighting.
We propose to study algorithms for the efficient construction of bitangents
for a curved environment, the organization of these bitangents into coherent
data structures, and the application of these structures to the computation
of visibility, shortest paths, and lighting.

The development of smooth curve and surface representations, such as Bezier 
curves and surfaces, NURBS, subdivision curves and surfaces, and implicit curves
and surfaces, has included a full treatment and analysis of tangent lines
and planes, with formulaes for their efficient and elegant construction,
such as tangent hodographs, tangents from the de Casteljau algorithm, and
tangent or normal masks for subdivision curves and surfaces.
However, there has not yet been an analogous complete treatment of bitangents.
This despite the integral importance of bitangents to visibility analysis
for efficient rendering and interactive interaction with a complicated scene,
to efficient motion of a robot through the environment, and to lighting from
non-point light sources.
This may be partially explained by the predominant use of polygonal and polyhedral 
environments in the early work on visibility, motion, and lighting,
for issues of cost and simplicity.
However, the area is maturing into the analysis of smooth enviroments
(smooth visibility analysis, smooth obstacles, smooth light sources)
and as this natural progression continues,
the understanding of bitangents of smooth environments will become
more and more important.

A 2-dimensional smooth environment is described by a collection of curves,
or by a collection of data points implicitly describing a collection of curves.
The underlying representation of the curve should be flexible:
Bezier curve, NURBS, subdivision curve, or implicit curve.
For example, a curve could be defined explicitly by a B-spline control polygon,
or implicitly as the $C^2$ cubic interpolating B-spline curve defined by 
the data points \cite{farintext}.
Or alternatively the curve could be defined explicitly by a Chaikin curve,
or implicitly as the Dyn-Levin-Gregory subdivision curve interpolating the
data points.
Similarly for a 3-dimensional smooth environment except the curves are replaced
by surfaces.

\clearpage

\Heading{APPLICATIONS}

\section{Smooth Visibility Analysis}
% \subsection{Silhouettes, Visual Events and Visibility Skeletons}

% see 672/silhbiblio.tex

\section{Smooth Lighting}

\section{Smooth Motion}
% \subsection{Visibility graphs}

\section{Motion planning in Orientation Space}

\section{Stable Placement of Objects}

\Heading{TOOLS}

\section{Bitangents}
\subsection{Bezier curves}

% See dual.tex, laptop notes, folder notes.

\subsection{Chaikin and subdivision curves}

\subsection{Bezier surfaces}

\subsection{On $S^3$ and other surfaces}

\section{Tritangents for Stable Object Placement}

\section{Computation with Developable surfaces}
% manipulation of developable surfaces (e.g., ruled surface intersection)
\subsection{Unrolling a ruled surface}
\subsection{Intersecting ruled surfaces}

\section{Synergistic Activities}

\begin{itemize}
\item Applet demos and tutorials.
\item Graphics camp.
\end{itemize}

\end{document}
