\documentclass[12pt]{article}
\usepackage{latex8}
\usepackage{times}
\usepackage{epsfig}
\input{header}

\newif\ifTalk
\Talktrue
\newif\ifJournal
\Journaltrue
\newif\ifFuture		% issues that are useful for future papers
\Futurefalse

\newcommand{\plucker}{Pl\"{u}cker\ }
\newcommand{\tang}{tangential curve\ }
\newcommand{\tangs}{tangential curves\ }
\newcommand{\Tang}{Tangential curve\ }
\newcommand{\atang}{tangential a-curve\ }
\newcommand{\btang}{tangential b-curve\ }
\newcommand{\ctang}{tangential c-curve\ }
\newcommand{\atangs}{tangential a-curves\ }
\newcommand{\btangs}{tangential b-curves\ }
\newcommand{\ctangs}{tangential c-curves\ }
\newcommand{\acurve}{a-curve\ }
\newcommand{\bcurve}{b-curve\ }
\newcommand{\ccurve}{c-curve\ }
\SingleSpace
\setlength{\headsep}{.5in}
	% \setlength{\oddsidemargin}{0pt}
	% \setlength{\topmargin}{-.2in}	% should be 0pt for 1in
\setlength{\textheight}{8in}
	% \setlength{\textwidth}{6.5in}
	% \setlength{\columnsep}{5mm}	% width of gutter between columns
\markright{Bitangency: \today \hfill}
\pagestyle{myheadings}

% -----------------------------------------------------------------------------

\title{Bitangency}
\author{John K. Johnstone\\
	Geometric Modeling Lab\\
	Computer and Information Sciences\\
	The University of Alabama at Birmingham\\
	University Station, Birmingham, AL 35294}

\begin{document}
\maketitle

{\bf Dagstuhl}
\begin{itemize}
\item	bitangency 
\begin{itemize}
\item	importance of bitangency [intro to SMI paper and opening slides of Dagstuhl]
\item   computing bitangents of parametric curves (rigorous comparison of tangential curve and implicit solution for parametric curves)
\end{itemize}
\item   {\bf applying bitangency: smooth convex hull by smooth giftwrapping}
\item   applying bitangency: smooth visibility graph
\begin{itemize}
\item 	Welzl improvement
\item   culling to invisible bitangents
\item   analysis of reduced complexity and increased precision 
	compared to polygonal visibility graph
\end{itemize}
\item 	applying bitangency: smooth kernel
\end{itemize}

{\bf WSCG}
\begin{itemize}
\item Computing bitangents of Chaikin curves by lazy subdivision
\end{itemize}

{\bf CAGD}
\begin{itemize}
\item The tangential surface
\end{itemize}

{\bf Graphics conference with fall submission, like Graphics Interface}
\begin{itemize}
\item Smooth umbra
\end{itemize}


\clearpage

% -----------------------------------------------------------------------------

\tableofcontents

\clearpage

\listoffigures

\clearpage

% -----------------------------------------------------------------------------

\section{A detailed discussion of bitangency}
\label{sec:bitangency}

The bitangent defines a visual event.

\ifTalk
\begin{rmk}
Where do you throw the ball so that it can be caught by another curve?
A bitangent indicates where the ball should be thrown from one curve 
so that it can be caught by a second curve.
Point/curve: Where do you shoot to hit the target (as you run around a track)?
Where does the spotlight illuminate a given spot?
\end{rmk}
\fi

The bitangents of two curves are at the heart of many problems
involving visibility.
% are fundamental in many geometric applications, particularly those that involve visibility.
Consider a collection of closed curves in the plane, interpreted as 
obstacles for a robot.
The bitangents of these curves define the limits of visibility,
and consequently the shortest paths that can be travelled by the robot
(Figure~\ref{fig:applications}).
If we instead interpret the collection of closed curves as 
an area light source and occluders,
the bitangents define the boundaries of the shadows (umbra and 
penumbra) that are cast by the light (Figure~\ref{fig:appl2}).
Common tangents are also needed for basic geometric computations with curves,
such as the convex hull.
\cite{parida95} mentions many more applications, including font design.
% bin-packing, and compaction.

Visibility complex's use of bitangents (Durand, Petitjean).

\begin{figure}
\caption{Bitangents for shortest paths}
\label{fig:applications}
% svgraph vg4b.rawctr, path only
% jjdush.gif
\end{figure}

\begin{figure}
\caption{Bitangents for lighting from an area light source}
\label{fig:appl2}
% svgraph vg4b.rawctr, visible bitangents on, source/destination off
% jjdulight.gif
\end{figure}

The bitangency problem is truly an intersection problem,
% The bitangency problem can be naturally posed as an intersection problem.
% The problem of bitangency is naturally posed as a problem of intersection.
since a bitangent is simultaneously a tangent of two curves.
% After all, a bitangent is a line that is simultaneously a tangent of two curves.
The standard solution reduces bitangency to the intersection of implicit curves.
We explore an alternate reduction, 
to the intersection of parametric Bezier curves.
% In this paper, we reduce the problem instead to the intersection
% of two parametric Bezier curves.
The robustness, efficiency, and simplicity of Bezier intersection
make this an attractive solution.

Let $C(s) = (c_x(s),c_y(s))$ and $D(t) = (d_x(t),d_y(t))$ be 
plane curves of degree $n$.
The standard solution \cite{bajajkim87,parida95} argues as follows.
% reduces to the intersection of two implicit curves.\footnote{An implicit 
%	curve in the plane is a curve represented as the zero set
%	of a bivariate polynomial, $f(x,y)=0$.}
Since a line between $C(s)$ and $D(t)$ is a bitangent if and only if 
it is orthogonal to the normal at both $C(s)$ and $D(t)$,
a bitangent is a parameter pair $(s,t)$ satisfying the following
system of equations:
%
\[
\begin{tabular}{c}
$(c_x(s) - d_x(t), c_y(s) - d_y(t)) \cdot (-c_y'(s), c_x'(s)) = 0$\\
$(c_x(s) - d_x(t), c_y(s) - d_y(t)) \cdot (-d_y'(t), d_x'(t)) = 0$
\end{tabular}
\]
%
This is the intersection of two implicit curves of degree $2n-1$
in parameter space:
\[
\begin{tabular}{c}
$f(s,t)=0$\\
$g(s,t)=0$
\end{tabular}
\]

\Comment{
A similar solution works for two
implicit curves, generating a more expensive system of 4 equations,
or for an implicit curve and a parametric curve, generating 3 equations
(see \cite{bajajkim87}).
}

% In this paper, we reduce the problem instead to the simpler intersection
% of two parametric Bezier curves.
In our solution, we interpret the tangent space of a curve as a family of lines
(Figure~\ref{fig:linefamily}),
and a bitangent as an intersection of the line families of two curves.
\Comment{
	This interpretation of bitangency as intersection
	will be a central theme of the paper.
}
To simplify the intersection of line families,
we map to a dual space. % where intersection is more natural.
By encoding a tangent line as a point in a dual space,
a family of tangent lines becomes a parametric curve in the dual space
(Figure~\ref{fig:1dual}).
The computation of bitangents then reduces to the intersection of 
curves in dual space (Figure~\ref{fig:biTang}).
The representation of tangent space as a rational Bezier curve in dual space
(Section~\ref{sec:bez}) is a development of independent interest.
It can be a useful tool for the analysis of tangent spaces.

% -----------------------------------------------------------------------------

\subsection{Bitangency as intersection: the final algorithm}

In many applications (e.g., motion planning and lighting),
visible bitangents are the only bitangents of interest,
and an extra filtering step should be added to the algorithm (Figure~\ref{fig:appl2}).
A bitangent is visible if its two points of tangency
are visible (i.e., the line segment connecting them
has no interior intersections with any obstacles).
Since a botangent is by definition close to the 
obstacle as it approaches it, and the intersection algorithm is approximate,
care must be taken in evaluating visibility.
% since it is easy to mistakenly find an apparent 
% intersection of the bitangent with an obstacle near the point of tangency,

Rather than using duality, the following solution is preferable
	to map the intersections back to bitangents. 
	We record each intersection as a parameter pair $(s,t)$,
	the parameter values of the intersection point with respect to each curve.
	The two endpoints $C(s)$ and $D(t)$ of the bitangent are then immediately 
	available.
	These endpoints are necessary for testing if a bitangent is visible.

The bitangents of a single curve can also be computed using this algorithm
(Figure~\ref{fig:eg4}).
The only difference is that $C$ and $D$ are the same curve,
and the intersection of step (3) is self-intersection \cite{lasser89}.
Figure~\ref{fig:eg4} shows the computation of the bitangents of a single curve.
The curves and their tangential curve systems are the same as in Figure~\ref{fig:eg2}:
we have just zoomed in on the tangential curves to better show detail
near the intersections.

\begin{figure*}
\caption{Bitangents of a single curve}
% dual ob1.rawctr, self-intersections and all bitangents between single curve
% jjdusingle.gif
\label{fig:eg4}
\end{figure*}

% -----------------------------------------------------------------------------

\subsection{Curve intersection}
\label{sec:intersect}

Possible interpretation: 
everybody knows about intersection: an insult to include it.

Once our problem has been translated to dual space by computing tangential curves,
it reduces to curve intersection, 
a problem that has received much attention and is well understood \cite{sederberg86}.
	% This is a major advantage of the tangential curve solution to bitangency.
The standard solution to intersection is a divide-and-conquer approach using
curve subdivision and the convex hull property of Bezier curves:
if the bounding boxes of the two curves intersect, subdivide each curve
and recursively intersect the four subsegments,
halting when the bounding box is sufficiently small 
(the curve is sufficiently straight) 
to compute the intersection point directly.
Sederberg reports that this is the most stable of the most popular
intersection algorithms.
\Comment{
% the less details about intersection, the better: well understood
A bounding box can be a convex hull or, for efficiency, 
an axis-aligned box defined by the extremal coordinates of the control points.
It turns out that the axis-aligned box, although larger, yields a more
efficient intersection algorithm,
since the convex hull is more difficult to compute.
}

Self-intersections of a Bezier curve (for the bitangents of a single curve)
can be computed using the method of Lasser \cite{lasser89},
which also uses subdivision, with some added subtlety
(for example, two neighbouring
segments of a curve will necessarily have an intersection at their
common boundary that must be ignored).

\Comment{
Elaboration on self-intersections: cannot have self-intersection
within a Bezier curve segment unless its control polygon has a self-intersection
(explain why using variation-diminishing property?)
so self-intersections are computed entirely the same as intersection
of two curves, where both curves are the same.
For example, the self-intersections of the Bezier spline with segments
$S_0,\ldots,S_k$ are the union of the intersections of $S_i$ and $S_j$ for $i \neq j$,
just as the intersections of the Bezier spline $C$ with segments
$C_0,\ldots,C_m$ with the Bezier spline $D$ with segments 
$D_0,\ldots,D_n$ are the union of the intersections of $C_i$ and $D_j$ for $i \neq j$.
}

\Comment{
We shall compute intersections of dual curves in 2-space, 
rather than projective 2-space.
Fortunately, the intersections in 2-space are equivalent to the
intersections in projective 2-space.
(This may be a well-known fact of projective space.)
Consider an intersection in 2-space: $(\frac{a_1}{c_1}, \frac{b_1}{c_1})
= (\frac{a_2}{c_2}, \frac{b_2}{c_2})$.
Then, in projective space, $(a_1,b_1,c_1) = \frac{c_1}{c_2} (a_2,b_2,c_2)$,
so the associated lines in projective space are also equivalent and 
form an intersection.
Conversely, if two 'points' (lines) in projective space are equivalent
(an intersection) then the associated points in projective space
are also equivalent: if $(a_1,b_1,c_1) = k(a_2,b_2,c_2)$ ($k \neq 0$),
then $(\frac{a_1}{c_1}, \frac{b_1}{c_1}) = (\frac{ka_2}{kc_2}, \frac{kb_2}{kc_2})
 = (\frac{a_2}{c_2}, \frac{b_2}{c_2})$.
}

% ---------------------------------------------------------------------------

\Comment{
IT WILL BE DIFFICULT TO IMPLEMENT: MUST TRANSLATE EACH PIECE OF THE SPLINE
INDIVIDUALLY, BUILD POWER-BASE POLYNOMIALS, COMPUTE RESULTANTS, ETC.
THE ADVANTAGES OF THE PARAMETRIC SOLUTION ARE CLEAR WITHOUT FURTHER ANALYSIS.
Comparison of execution times for our method (parametric curves in dual space)
and the classical method (implicit curves in parametric space)
in Table~\ref{tab:speedBajaj}.

\begin{table}
\begin{tabular}{|c|c|c|}  	\hline
Data set & Time, old solution (seconds) & Time, our new solution\\ \hline
1 & ? & ? \\ \hline
2 & ? & ? \\ \hline
3 & ? & ? \\ \hline
\end{tabular}
\caption{Improved efficiency of our solution}
\label{tab:speedBajaj}
\end{table}

OVERKILL (ALTHOUGH SPEED DIFFERENCE IS NOTICEABLE); PREFER PITHINESS.
The increase in efficiency gained by clipping at diagonal tangents
is illustrated in Table~\ref{tab:speed}.
{Also show figures of difference, like in jones.rgb?}
% Comparison of execution time of dual and dual-00Aug23preDiag:
% clipped by diagonals and clipped only about zeros
% (i.e., computing all intersections)

\begin{table}
\begin{tabular}{|c|c|c|}  	\hline
Data set & Time, clipping immediately around zeroes (seconds) & Time, clipping at diagonal tangents\\ \hline
1 & ? & ? \\ \hline
2 & ? & ? \\ \hline
3 & ? & ? \\ \hline
\end{tabular}
\caption{Improved efficiency through clipping}
\label{tab:speed}
\end{table}
}

% -----------------------------------------------------------------------------

\subsection{Related work on bitangency}
\label{sec:prevcommon}

\begin{itemize}
\item head-on comparison with implicit solution (implement implicit?)
\item complexity analysis
\end{itemize}

% MAKE IT SIMPLE: INTERSECTION OF PARAMETRIC RATHER THAN INTERSECTION OF IMPLICIT;
% SIMPLE SUBDIVISION OF PARAMETRIC INTERSECTION RATHER THAN COMPLICATED SUBDIVISION OF PARIDA.

In the introduction, we saw that the classical solution to the bitangents
of the parametric curves $C$ and $D$ of degree $n$
reduces to the intersection of two implicit curves $f(s,t)=0$ and $g(s,t)=0$
of degree $2n-1$.
We would like to compare this implicit solution to our parametric one.
Suppose that $C$ and $D$ are Bezier splines.
Unfortunately, the implicit curve solution does not take advantage
of the Bezier structure of the original curves,
and is actually undermined by it.
$f$ and $g$ are piecewise polynomial and 
we must reduce them to their constituent polynomials in order
to apply the basic solution (see below), resulting
in many polynomial intersection pairs.

The solution for a pair of polynomials is itself rather expensive.
Suppose that we have decomposed the implicit intersection into
its constituent polynomial implicit intersections.
The solution for polynomials $f$ and $g$ is as follows.
We must compute and solve the resultant of $f$ and $g$
with respect to both $s$ and $t$, which isolates the $O(4n^2)$ $s$- and 
$t$-coordinates of the intersections.
We must then evaluate $f(s,t)$ and $g(s,t)$ for all $O(16n^4)$ parameter pairs
to filter out the true intersections.
Each evaluation of $f$ and $g$ involves the evaluation of 
the two curves $C(s)$ and $D(t)$ and their hodographs $C'(s)$ and $D'(t)$.
% Therefore, this is a time-consuming computation.
% This computation depends on the fact that $f$ and $g$ are polynomial
% (for the resultant).
\Comment{
The $s$-coordinates $\{s_i\}$ of the intersections are found by computing
the Sylvester resultant of the two polynomials with respect to $s$, $R_s(f,g)$,
and solving this univariate equation of degree $(2n-1)^2$ in $s$.
The $t$-coordinates $\{t_j\}$ of the intersections are found similarly, 
by solving $R_t(f,g)=0$.
This generates a superset $\{(s_i,t_j)\}$ of the intersections,
which must be filtered down to the intersections by evaluating
$f(s_i,t_j)$ and $g(s_i,t_j)$ for all pairs, saving those pairs that
satisfy both equations.
}

In contrast, the reduction of bitangency to Bezier curve intersection
takes advantage of the efficiency and stability of operations 
on Bezier curves (e.g., divide-and-conquer by subdivision and the convex-hull
property).
The clipping of the Bezier curves (all segments lie inside the strip 
$x \in [-1,1]$) also contributes strongly to efficient intersection.
\Comment{
There are also advantages to working in a dual space where points directly correspond
to tangent lines (the points of the curves $f(s,t)$ and $g(s,t)$ in parameter space have
no direct correspondence with tangents).
}

% (and require translation to a polynomial basis
% from the Bernstein basis, for the calculation of resultants).
% ({\bf Or is there a resultant of Bezier polynomials?})

Another solution to bitangents has been proposed by 
Parida and Mudur \cite{parida95}, using a geometric divide-and-conquer approach.
Their solution works as follows.
The curves are first decomposed into 'C-shaped curves', convex segments
monotone in tangent direction.
Each pair of C-curves, one from each curve, is compared to determine
if the pair can define a bitangent, as follows.
The pair is first reparameterized and clipped so that the C-curves
share the same parameter interval and have the same tangent directions at the endpoints. 
If certain rejection criteria involving the tangent range are satisfied,
the pair is rejected.
If the two C-curves are pseudo-linear, a test for a bitangent is
directly applied.
Otherwise, both C-curves are subdivided at their shoulder points (the point
whose tangent is parallel to the chord of the endpoints) and the process
is repeated recursively.
The subdivision involved in this solution is more complicated than the
simple midpoint subdivision of Bezier intersection,
requiring the calculation of a shoulder point at each subdivision
and reparameterization.
Moreover, an initial decomposition into C-curves is expensive, and 
there is more subdivision.
% A great deal of decomposition would be required for complicated examples
% such as Figure~\ref{fig:result1234} and \ref{fig:result6789},
% which are much more complicated than the examples in \cite{parida95}.

% Moreover, it does not lead to a generalizable solution.

We observe that \cite{sederberg90} solves a special case of the 
bitangent problem, showing how to compute the points of tangency of two plane Bezier curves
(i.e., an intersection point of the curves where they share the same tangent).

% -----------------------------------------------------------------------------

\section{Tangents through a point}

% -----------------------------------------------------------------------------

\subsection{Relationship to pole/polar theory}

% -----------------------------------------------------------------------------

\section{The smooth visibility graph}

% -----------------------------------------------------------------------------

\bibliographystyle{plain}
\begin{thebibliography}{99}	% {Lozano-Perez 83}

\bibitem{bajajkim87}
Bajaj, C. and M.-S. Kim (1987)
Convex hull of objects bounded by algebraic curves.
Technical Report CSD-TR-697, Computer Science, Purdue University.

\bibitem{jj01b}
Johnstone, J. (2001)
Smooth visibility from a point.
39th Annual ACM Southeast Conference.
Athens, Georgia, ACM Press, 296--302.

\ifJournal
\bibitem{kim89}
Kim, M.-S. (1989)
Motion Planning with Geometric Models.
Ph.D. thesis, Computer Science, Purdue University.
\fi

\bibitem{lasser89}
Lasser, D. (1989)
Calculating the self-intersections of Bezier curves.
Computers in Industry 12, 259--268.

\bibitem{parida95}
Parida, L. and S. Mudur (1995)
Common tangents to planar parametric curves: a geometric solution.
Computer-Aided Design 27(1), 41--47.

\ifJournal
\bibitem{rockwood90}
Rockwood, A. (1990)
Accurate display of tensor product isosurfaces.
Visualization '90, 353--360.
\fi

\bibitem{sederberg90}
Sederberg, T. and T. Nishita (1990)
Curve intersection using Bezier clipping.
Computer Aided Design 22(9), 538--549.

\end{thebibliography}

% -----------------------------------------------------------------------------

\section{Appendix}

\Comment{
\begin{defn2}
A {\bf bitangent} of the curves C and D is a line that is tangent to both C and D.
We do not distinguish whether the line is tangent at one or more points
of each curve.
A {\bf bitangent of a single curve} C is a line that is tangent to C at two or more
distinct points.
\end{defn2}
}

% We review the relationship of projective space to rational Bezier curves.

% \begin{defn2}
% {\bf Projective 2-space} $P^2$ is the space 
% $\{(x_1,x_2,x_3) : x_i \in \Re, \mbox{ not all zero}\}$
% under the equivalence relation $(x_1,x_2,x_3) = k(x_1,x_2,x_3),\ k \neq 0 \in \Re$.
% \end{defn2}

\Comment{
\begin{figure}
\vspace{8in}
\special{psfile=/usr/people/jj/modelTR/6-dual/img/alg14.ps}
\caption{Our algorithm}
\label{fig14}
% uses dual-00Aug23preDiag ob2.rawctr, a-curves only
% sgi2ps -C w -O alg14.ps -z .9 -N 2 alg1.rgb alg2.rgb alg3.rgb alg4.rgb
% The steps of the basic algorithm are illustrated in Figure~\ref{fig14},
% where we actually use the \atangs of Section~\ref{sec:infinity}.
\end{figure}
}
% ADD THE FOLLOWING CAPTIONS MANUALLY:
% Two curves (left) and their \tangs (right)
% Intersections of \tangs in dual space
% Associated bitangents in primal space
% Visible bitangents

\Comment{
\twocolumn

\begin{figure}
\vspace{2in}
\special{psfile=/usr/people/jj/modelTR/6-dual/img/alg1.ps hoffset=-50}
\caption{Two curves (left) and their \tangs (right)}
% sgi2ps -C w -z .6 -O alg1.ps alg1.rgb
% tops alg1.rgb -m 4 1.5 > alg1.ps
\label{fig:alg1}
\end{figure}

\begin{figure}
\vspace{2in}
\special{psfile=/usr/people/jj/modelTR/6-dual/img/alg2.ps hoffset=-50}
\caption{Intersections of \tangs in dual space}
% sgi2ps -C w -z .3 -O alg2.ps alg2.rgb
\label{fig:alg2}
\end{figure}

\begin{figure}
\vspace{2in}
\special{psfile=/usr/people/jj/modelTR/6-dual/img/alg3.ps hoffset=40}
\caption{Associated bitangents in primal space}
% sgi2ps -C w -z .3 -O alg3.ps alg3.rgb
\label{fig:alg3}
\end{figure}

\begin{figure}
\vspace{2in}
\special{psfile=/usr/people/jj/modelTR/6-dual/img/alg4.ps hoffset=40}
\caption{Visible bitangents}
% sgi2ps -C w -z .3 -O alg4.ps alg4.rgb
\label{fig:alg4}
\end{figure}

\clearpage
\onecolumn
}

\Comment{
\begin{figure}
\vspace{2in}
\caption{Switching from (a) a single tangential curve to (b) two clipped tangential curves}
\end{figure}
}

\begin{figure}
% \vspace{8in}
% \special{giffile=/usr/people/jj/modelTR/6-dual/img/result1.gif}
\caption{Bitangents}
\label{fig:result1234}
% sgi2ps -C w -O result1234.ps -z .9 -N 2,2 result1.rgb result2.rgb result3.rgb result5.rgb
% dual ob1.rawctr
\end{figure}

\begin{figure}
% \vspace{8in}
% \special{psfile=/usr/people/jj/modelTR/6-dual/img/result6789.ps}
\caption{More bitangents}
\label{fig:result6789}
% sgi2ps -C w -O result6789.ps -z .9 -N 2,2 result6.rgb result7.rgb result8.rgb result9.rgb
% 6-7: dual -p -.2 .05 ob2.rawctr
% 7-8: dual -p 0 -1 ob3.rawctr
\end{figure}

{\bf Build lots of applets to advertise this research,
including tangential curves, duality, visibility graph.}

\subsection{Visibility graphs of curved environments and shortest paths}

\subsection{Shading with curved area lights}

\subsection{Curved Minkowski sums, for curved robots}

(The following is a reasonable topic in the context of my work on 
swept surfaces and envelopes, but unrelated to tangential computation
and curves on surfaces.)
Curved extension of Minkowski sum (see Bajaj and Kim, Generation of Configuration
Space Obstacles: The Case of Moving Algebraic Curves
and Compliant Motion Planning with Geometric Models).
That is, for a non-point robot, the robot must first be shrunk to a point
and the obstacles expanded by their Minkowski sum with the robot
in order to apply the visibility graph algorithm.
Typically the robot and obstacles have been polygonal.
(A circular robot has also been dealt with by Yap.)
We have shown how to deal with curved obstacles.
Now we want to deal with a curved robot.

\subsection{Other tangential relationships from \tangs}

are tangents of the dual curve = points of the original curve? (test empirically)
proof: bitangents of dual curves must be intersections in original space, 
	by symmetry, thus tangents of dual must be points of original curve
a way to compute the envelope of a family of lines:
	translate family of lines to dual space, where it becomes a curve;
	compute the tangents of this dual curve (hodograph), 
	whose duals become the envelope curve in the original space
	[is this the striction curve of the ruled surface?]
	not interesting in 2-space, but analogue in 3-space would be interesting:
	computing envelope of planes
	
Hypothesis: Consider a plane moving through 3-space.
	Its image in dual space is a curve (in 3-space).
	The tangents of this space curve are lines in dual space.
	I believe that as points dualize to planes and planes to points,
	lines dualize to lines.
	If so, the tangent lines of the space curve in dual space
	(which form a ruled surface, actually a tangent developable)
	dualize to a ruled surface in the primal space.
	Could this be the envelope of the moving planes?

\subsection{Common tangents of curves on surfaces}

See paper.tex in modelTR/6-dual/onS2 directory

\subsection{Common tangents on \Sn{2}}

See paper.tex in modelTR/6-dual/onS2 directory

\subsection{Common tangents on \Sn{3}}

ditto.

\subsection{Common tangents on developable surfaces}

A geodesic on a developable surface is also tractable:
it is a rolled-up line.


\subsection{Dual curves of space curves}

We have shown how to compute the dual of a plane curve and a surface.
What about a space curve?
Unfortunately, duals only exist for objects of codimension 1.
(Define codimension.)
The tangent of a space curve is a line in 3-space, which can no longer
be expressed as a single linear equation since it is no longer a hyperplane.
Consequently, the dual of a space curve tangent, and thus of a space curve,
is undefined.

\subsection{Common tangents of two subdivision curves}

\subsection{Scraps}

The tangent space of a curve is a smooth one-dimensional family of lines.
A line can be represented by a point in \plucker space or a point in dual space,
and a continuous one-dimensional line family becomes a curve 
in the \plucker or dual space (henceforth, \plucker or dual curve).
We are interested in the representation of curve tangent spaces as
\plucker or dual curves.
We are also interested in the computation of bitangents between
plane curves through the intersection of \plucker or dual curves.

%%%%%%%%%%%%%%%%%%%%%%%%

[To get around the instability of points at infinity,
one may use two interpretations of the curve in projective space:
the first with $x_i$ as the projective coordinate and the second
with $x_j$ ($i \neq j$) as the projective coordinate.
In this scheme, a one-dimensional family of lines would be represented
by two curves in image space (e.g., dual space or Plucker space),
which are theoretically redundant but, in practice, map different parts
of the family to infinity and thus together allow robust treatment of all parts
of the family, as long as wrong solutions arising due to instabilities
near infinity can be detected.
This detection of wrong solutions is possible with bitangents, 
since it is simple to determine
if a candidate tangent is actually a bitangent.

This leads to the following algorithm:
\begin{enumerate}
\item Intersect dual curves using $x_3$ as projective coordinate.
\item Intersect dual curves using $x_2$ as projective coordinate.
\item Use all intersections as candidates.  Reject solutions that
	do not yield bitangents.
\end{enumerate}
]

[Luckily, we have a choice of which coordinate to choose as the projective one
and not all of these choices can have a zero at the same point.
Thus, for any given point, we can choose the projective coordinate to be
a nonzero one.
Or in practice, we can compute all three curves, using all choices of projective
coordinate, and perform the computations with respect to all these curves.]

% -----------------------------------------------------------------------------

\section{Related papers}

\begin{itemize}
\item Web site demos of the tangential curve system
\item Tangential surfaces (teapot)
\item The convex hull (and visibility graph?) of smooth curves
\begin{itemize}
\item culling invisible bitangents
\item efficient calculation a la Welzl
\end{itemize}
\item The kernel of a curve and surface
\begin{itemize}
\item computing a point of the kernel (so that it can be made the origin
		of the duality)
\item computing the convex hull in dual space
\item dualizing the hull back to primal space
\end{itemize}
\item On the computation of bitangency
\begin{itemize}
\item	2nd half of SMI paper
\item   concentration on applications of visibility and its computation
\item   visibility graph discussion
\item   special case of tangents through a point and pole/polar discussion
\end{itemize}
\item Chaikin tangential curves (+ Catmull-Clark tangential surfaces?)
\item Bitangent developables of ringed surfaces (with Myung-Soo)
\item Smooth shadows of curves
\end{itemize}



\end{document}
