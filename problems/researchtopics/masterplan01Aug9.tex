FUTURE PAPERS:

journal version of common tangent paper:
0) proof that our duality is a duality 
	and comparison with other dualities (e.g., pictures)
1) tangents through a point and polars
2) visibility graph construction, including clipping of invisible tangents
	and possibly an incorporation of Welzl's ideas
3) shortest paths amongst curved obstacles discussion
4) comparison with polygonal visibility graphs and discussion of smooth advantages
	- could rip 3 and 4 out into an R&A article if we can elaborate
	  sufficiently on 3-4, as well as elaboration of 2 for fullness of the present paper
5) additional references: e.g., Welzl and Edelsbrunner
6) implement implicit solution and analytically compare execution speed
	with clipped Bezier (problem: have to implement Nishita and Sederberg,
	or Sherbrooke and Patrikalakis)
7) elaboration of umbra and penumbra (or this might be extended into a
	separate paper)

--------------------------------------------------

journal version of bisilhouette paper could include this:
Title: The bisilhouette of two surfaces
1) material on rational Bezier representation of tangential surfaces, 
	and removing negative weights
2) some of the missing proofs
3) bitangent developables of the same surface, for convex hull
4) trimming common silhouettes of concave shapes: not interested in
	bitangent planes between points that cannot see each other,
	either because of a concavity within A or B,
	or because of an intervening object C
	
--------------------------------------------------
	
paper incorporating subdivision curves and surfaces, rather than Bezier curves
 - bitangent mask for two subdivision curves
 - bisilhouette of two subdivision surfaces as a subdivision surface
 - SMI2002
 
--------------------------------------------------

NSF: visibility analysis in dual space
ONR: smooth motion (shortest path and quaternion spline); visibility testbed
EPSCOR: ONR/NSF piggyback
 
--------------------------------------------------

bitangent in 2-space (Bezier, Chaikin, Sederberg algebraic)
bitangent in 3-space (Bezier, Loop)
incorporation in data structure (visibility graph, visibility complex)
application (lighting, motion, visibility, morphing)

CAGD			     bitangent of Bezier curve
CAGD			     bisilhouette
Shape Modeling		     bisilhouette
Graphics Interface	     morphing/mating 2 curves
Eurographics		     bitangents of Chaikin subdivision curves

--------------------------------------------------

Solid Modeling 		     tritangent planes of support
Computational Geometry	     smooth visibility graph
Dagstuhl		     bitangents
ACM Southeast		     circle visibility graph
			     comparison of smooth/polygonal visibility graph
Eurographics 		     bitangents on S3, shortest paths in orientation space
			     shadow lines on a surface; smooth soft shadows, especially penumbra
Miscellany: 		     unrolling the bisilhouette (for motion)
			     envelopes in dual space
 
--------------------------------------------------

- Java testbed

- tritangent planes of support (see Koenderink)
  for stably placing object down (e.g., robotics)
  - this paper finds the bitangent planes of support, which are not stable
  
- lighting with the common silhouette: umbra, penumbra

- shortest path motion
 	A natural solution may result from 'unrolling' the common silhouette.
	This may be feasible since it is developable.

- bitangents of subdivision curves and surfaces

- smooth visibility complex (but see Baciu in PG2000, p. 105)

- aspect graphs
	The construction of an aspect graph for an environment of objects
	defined by smooth surfaces (not polyhedra)
	requires the bitangents of the objects.
	CHECK SMOOTH ASPECT GRAPH LITERATURE.
	DON'T WANT TO POSE THIS AS A SOLUTION TO ASPECT GRAPHS.
	The common tangents help define the locations of changes of aspect:
	the qualitative appearance of the objects has an abrupt change
	since the occlusion of the objects changes.
	
	The aspect changes as the eye crosses the common silhouette.
	This makes the construction of the common silhouette
	a crucial operation for visibility analysis.

- bitangents *on* a surface

- do we have anything to say about silhouette (visibility from a 
  point to a surface)?

- convex hull of smooth surfaces (uses bitangent developables)
	- see Koenderink 'Solid Shape' on support planes

- computing the envelope of a 1-parameter family of planes.
\begin{itemize}
\item plane family in primal space $\rightarrow$ curve in dual space
\item tangent lines of curve in dual space $\rightarrow$ 
      generators of envelope
\item directrix curves of envelope can be chosen as duals of two continuous
      plane families along the curve in dual space, 
      such as two plane families defined by the Frenet frame along the curve:
      planes orthogonal to tangent and planes orthogonal to normal
      (or planes orthogonal to binormal).
\end{itemize}

Given a one-dimensional family of planes in primal space 
associated with a curve in dual space, we (think we) can find the envelope
of the planes through the tangent space of the dual curve.

Possible method: we want a family of tangent planes along the intersection
	curve.  A plane is dictated by a normal (plus a point).
	We could compute the normals of the tangential surfaces at every
	discrete intersection point computed by Sederberg.
	Then interpolate the normals to get a smooth approximating family of planes.
	(This may involve interpolating a curve on the sphere, like quaternion
	splines, if we use or need unit normals.
	Try the simpler interpolation of non-unit normals first, though.)

\begin{enumerate}
\item Compute normals at all intersection points, on both $A^*$ and $B^*$.
\item Interpolate the 4-dimensional points
	$(N_i, -P_i \cdot N_i)$ using standard interpolation in 4-space,
	say yielding $C^*(t)$ and $D^*(t)$.
	Don't normalize the normals.
\item Dualize $C^*$ and $D^*$ to curves.
\item Alternatively, dualize discrete sampling of tangent planes in dual space
	to points in primal space, and interpolate these points conventionally
	to yield directrix curve.
\end{enumerate}
Explanation of (4):
	For each of $A^*$ and $B^*$,
	interpolate the discrete tangent planes at the intersection points,
	producing a 4-dimensional implicit plane Bezier representation
	$(x_1(t), x_2(t), x_3(t), x_4(t))$ representing the family
	of tangent planes across the intersection curve.
	$(x_1,x_2,x_3)$ are an interpolation of the normals
	and $x_4$ is an interpolation of $-P \cdot N$.
	(Recall that $(X - P) \cdot N = 0$ is implicit equation of plane.)

Explanation of (5):
	$C$ and $D$ are the directrix curves of the common tangent surface.
	This is related to the dualization of planes to tangential surfaces
	in the first step, but with a one-dimensional,
	not two-dimensional, family of planes.
	
\section{Generalizations}

A generalization of the definition of dualism:
A hyperplane in $n$-space can be associated with a point in projective 
$n$-space, since the implicit equation of the hyperplane has $n+1$ coefficients 
and the point has $n+1$ coordinates.
It is appropriate for the point to be in projective space,
since any nonzero multiple of the hyperplane's equation represents the same plane.
%
\begin{defn2}
A {\bf hypersurface} in $n$-space is an $(n-1)$-manifold (a surface of codimension 1).
For example, curves in 2-space and surfaces in 3-space are hypersurfaces.
A {\bf hyperplane} is a linear hypersurface.
For example, lines in 2-space and planes in 3-space are hyperplanes.
\end{defn2}
%
\begin{defn2}
The hyperplane $a_1x_1+\cdots+a_nx_n+a_{n+1}=0$ % in $n$-space
and the point $(a_1,\ldots,a_{n+1}) \in P^n$ are {\bf duals}.
\end{defn2}
%
\begin{defn2}
The {\bf dual of a hypersurface} $C(t_1,\ldots,t_{n-1})$ in $n$-space is the
hypersurface \linebreak $C^*(t_1,\ldots,t_{n-1}) \subset P^n$ where the point 
$C^*(t_1,\ldots,t_{n-1})$ is the dual 
of the tangent hyperplane at $C(t_1,\ldots,t_{n-1})$. 
% \cite[p. 54]{hartshorne}
\end{defn2}


\Comment{
The full potential of the common tangent solution in dual space is realized 
with surfaces.
The dual space solution for common tangents between a point and curve is
comparable to the standard solution.
The dual space solution for common tangents between two curves 
offers many more advantages, but still involves the intersection of two curves,
which was the standard solution.
(The main difference is that the intersection is now of parametric Bezier curves
rather than implicit curves.)
However, the difference between the implicit and dual solutions for surfaces
are far more stark and obvious: rather than intersecting 2 surfaces
in 4-space, we can intersect 2 surfaces in 3-space.
}

