\documentstyle[12pt]{article} 
\newcommand{\DoubleSpace}{\edef\baselinestretch{1.4}\Large\normalsize}
\input{/rb/jj/texlib/pageformat-double.tex}
\setlength{\oddsidemargin}{-.25in}
\newcommand{\Comment}[1]{\relax}  % makes a "comment" (not expanded)
\begin{document}
\pagestyle{empty}

\centerline{\bf {\em CONFERENCES}}
\begin{table}[h]
\centering
\begin{tabular}{|l|l|l|}
\hline
Cornell Workshop & Sept. 23 (1 page abstract) & Quaternion curves \\
\hline
Solid Modeling '95 & Nov. 1 (full paper Nov. 15) & Quaternion curves (for swept surfaces) \\
\hline
Graphics Interface '95 & Oct. 31 & Quaternion curves (for animation) \\
\hline
CG Int'l '95 (Virtual Reality) & Dec. 1 & Cor/sag/trans reconstruction \\
\hline
Eurographics '95 & Jan. 9, 1995 & Cor/sag/trans reconstruction \\
\hline
\end{tabular}
\end{table}

\vspace{2in}

\centerline{\bf {\em GRANTS}}
\begin{table}[h]
\centering
\begin{tabular}{|l|l|l|}
\hline
NSF  & Swept surfaces & \\
\hline
NASA & Unstructured grid visualization & with Ken \\
\hline
Joint NSF & Reconstruction of cor/trans/sag & with Ken \\
\hline
Cardiology Consortium & 3D Modeling of the Heart & \\
\hline
Minority Female Fellowship (NSF?) & Medical informatics & Ken, Warren \\
\hline
\end{tabular}
\end{table}

\clearpage

\centerline{\bf {\em INTERDISCIPLINARY RESEARCH}}
\begin{table}[h]
\centering
\begin{tabular}{|l|l|l|}
\hline
Cardiology & Pohost,Blackwell & Surface models for wall thickness \\
& Coghlan, Singleton & Reconstruction of coronary arteries \\
& & Eigenvector-based triangulation \\
\hline
Virtual surgery & Grimes & \\
\hline
Neurosurgery & Guthrie, Fast, Freund & \\
\hline
Biomedical Engineering & Stokely & \\
\hline
\end{tabular}
\end{table}

\clearpage

\centerline{\em Geometric modeling}

Our present emphasis in geometric modeling  is on the swept surface.
Geometric modeling is the area of graphics concerned with the 
creation of models of solids for display and analysis.
Swept surfaces abound in machining, robot motion, ship/aircraft/auto design,
and animation.
Simple swept surfaces (e.g., extruded and revolute surfaces) are available 
in solid modeling systems, but arbitrary swept surfaces are not.
The first goal of this project is the full incorporation of a larger class of
swept surfaces (including at least the arbitrary sweep of simple curves
such as lines and circles) into solid models.
This requires concentration on surface intersection and other Boolean operations
on solids involving swept surfaces.
A second goal is the development of powerful tools for the creation and
analysis of objects sweeping through space, which demands work on 
representational issues, sweep generation, interference detection, volume
and other properties.
Current and past research includes work on rational Bezier 
representations of swept surfaces, animation keyframing, 
blending surfaces, ruled surfaces, 
canal surfaces, cyclides, and quadrics.

\vspace{2in}

\begin{table}[h]
\centering
\begin{tabular}{|l|l|l|l|}
\hline
Swept surfaces & Ruled surfaces &  & Journal final version \\
\hline
Swept surfaces & Quaternions: drawing on the sphere & Jimbo & Fall conference \\
\hline
Swept surfaces & c-r-o Bezier swept (esp. ringed) surfaces & Jimbo & Spring conference \\
\hline
Swept surfaces & Ruled-ruled (ringed/ringed) intersection & & Fall conference \\
\hline
\end{tabular}
\end{table}

\clearpage

\centerline{\em Biomedical Visualization and Image-guided Therapy}

The raw data of medicine is CT and MRI images.
Many areas---such as cardiology, neurosurgery, plastic surgery, 
and anatomy---can benefit from reconstruction of this CT/MRI data into
3D surface models.  These models can be used to aid diagnosis,
plan (before) surgery, and guide (during) surgery.
The focus of this project is the creation of accurate and measurable
3D models of anatomy from CT/MRI, using techniques from geometric
modeling and computer aided geometric design.

\vspace{2in}

\begin{table}[h]
\centering
\begin{tabular}{|l|l|l|l|}
\hline
Visualization & Reconstruction: cor/trans/sag Coons [brain atlas] & Ken & Spring conference \\
\hline
Visualization & Mensuration: arc length, surface area, volume & Stokely & \\
\hline
Visualization & Flexible models: quadric control nets [brain atlas] & & \\
\hline
Visualization & Area refinement of Bezier patch [reconstruction] & Woo/Chirikjian & \\
\hline
Visualization & Model based segmentation of liver & Ney/Berland?? & \\
\hline
\end{tabular}
\end{table}

\clearpage

\begin{table}[h]
\centering
\begin{tabular}{|l|l|l|l|}
\hline
Solid modeling & 2-line representation of hyperboloid & Shene & Fall journal \\
\hline
\end{tabular}
\end{table}

\clearpage

\begin{table}
\centering
\begin{tabular}{|l|l|l|}
\hline
Title		& Conference paper	& Journal paper 
	\\ \hline \hline
% ****************************************************
\Comment{
Sorting		& SIAM87	& SJOC (long!) \\ \hline
Coordinated Crawling & CGI89	& Visual Computer \\ \hline
Plane-Quadric Intersection  & - - - & Computers and Graphics \\ \hline 
Ringed/Cyclide & Stony, SIAM91	& CAGD \\ \hline  
	% (subm. 11/20/91, accepted 7/8/92)
Plane/Revolute & & Computers and Graphics \\ \hline
Blending of cones by cyclides & IMA & \\ \hline
Bisector: computing + curve-curve & & IMA \\ \hline
Natural Quadrics & ACM Solid Mod '91	& TOG \\ \hline
	% subm. 9/9/91, revised version submitted 6/9/93
	% to Rossignac	
}
%
Ruled Surface	& Math and Comp '89 & CAD \\ \hline
	% (subm. 9/10/91) 
	%  to John Woodwark
	% revised version sent to L. Piegl on 9/29/92
Distance Rep 	& DIMACS,Wayne	& Int.J.CG \\ \hline
\end{tabular}
\end{table}
% Ruled/ringed intersection & tech report & \\ \hline
% 16. sorting of Bezier (thesis methods) \\ \hline

\clearpage

\begin{itemize}
\item
{\bf Ruled surfaces}
\begin{itemize}
\item as the most primitive swept surface
\end{itemize}

\begin{table}[h]
\centering
\begin{tabular}{|l|}
%
\hline
{\bf Applications} \\
\hline
motion of linkages \\
wire EDM \\
tangent envelopes for medial axis \\
convex hull \\
family of offsets of a space curve \\
\hline
\end{tabular}
\end{table}

\clearpage

\item
{\bf General swept surfaces}

\begin{table}[h]
\centering
\begin{tabular}{|l|}
%
\hline
%When does a pencil contain a ruled? \\
Ruled, swept defn (pipe, artery, ...) \\
	% e.g., defn of artery for Glacov/Giddens project
	% 	defn of pipes for architectural design
Ruled, swept intersection \\
Cyclide extension \\
Cyclide patch defn and intersection \\
\hline
\end{tabular}
\end{table}

\begin{table}[h]
\centering
\begin{tabular}{|l|}
\hline
{\bf Envelopes} \\
intersection\\
convex hull \\
bitangents \\
interference \\
Bezier as envelope \\
optimal sweep \\
skeleton \\
generalized cylinder \\
ruled surface in pencil \\
spherical product, ruled, ringed \\
Voronoi diagram via upper envelope \\
\hline
\end{tabular}
\end{table}

\clearpage

\item
{\bf Flexible models}
\begin{itemize}
\item capable of deformation while maintaining geometric invariants
	(such as quadric surface, swept surface, etc.)
\item a.k.a. deformable model 
\item first attack: flexible quadric nets (try flexible ellipses even
	before that) % as in flexible ellipse making best match to
			% artery as it sweeps along slices
\item medical attack: anatomical models that are deformable to the 
	patient's model
\end{itemize}

\begin{table}[h]
\centering
\begin{tabular}{|l|}
%
\hline
%Get bounds on the cutter size (toroidal with min and max radius),\\
%\ \ that is, offsets that avoid bad (pole) curvatures. \\
% prefer tight bounds on cutter size since large radius of cutter
%	makes small changes in angle lead to large changes in radius
%	which is less robust (need to control angle carefully)
%\hline
When is a control net [a parameterization] a quadric?  \\
Given a quadric control net and a movement of one of its control points, \\
what movement of the other control points is necessary to maintain a
quadric? \\
% options: 1. algorithm for testing, with or without using 
% 			implicitization (resultants) (with is easy)
% 	   2. analog of invariants to implicit representation
%	   3. geometric construction on control points
% 
\hline
\end{tabular}
\end{table}

Efficient algorithms, and tractable surfaces, are very desirable.
However, they can only be used in special circumstances.
(Analogy, rational numbers are preferable to reals, but are sparse
and special.)
We want to understand the restrictions that must be put on a design
to allow tractable surfaces and efficient algorithms.
Thus, our goal is the development of restrictions for tractability.
Two exemplary results are the conditions for planar intersection
and the conditions for cyclide blending of cones.
A design can try to relax onto a simple special case nearby,
or we can view the conditions for the special case as predefined design
parameters (as in the above problem of definition of control
nets that begin quadric and remain quadric under dynamic online
movement of control points).

\clearpage

\item
{\bf Reconstruction}
\begin{itemize}
\item of anatomical models
\item new patch: a model that reflects all three slice directions 
	(sag,cor,trans), thus extending Coons patches
\item landmark extraction from anatomical models, for a small set
	of parameters to define the variations in an organ
	(tieing in with above flexible anatomical models):
	inflections, singularities, min curvatures, ...
\item quadric version of Kalvin's polyhedral anatomical solid models,
	for added conciseness and flexibility,
	possibly still based on winged edge and Euler ops 
\item computation of optimal correspondence between 2 slices
	(e.g., minimal area), given two parallel slices defined 
	as *curves* rather than point sets
\end{itemize}

\begin{table}[h]
\centering
\begin{tabular}{|l|}
%
\hline
Reconstruction from 3 slice directions \\
Optimal correspondence \\  % optical flow?
Surface area, volume \\ % problem: most curves and surfaces are not rectifiable
Concise deformable model \\
\hline
\end{tabular}
\end{table}

\item
{\bf Integral properties} (with Tony and Greg)
\end{itemize}

\begin{table}[h]
\centering
\begin{tabular}{|l|}
%
\hline
Diff geom: Gauss curvature, par/umb, bitangents \\
\hline
\end{tabular}
\end{table}

\clearpage

\end{document}

Given a (c,r,o) ring surface, give the equivalent (or approximate)
Bezier surface.

For clipping a ring surface:
Given a Bezier circle, clip it at its intersection with a plane.
Given a Bezier ring surface, clip it at its intersection with a plane.

Given a surface defined as a sweeping envelope of a sphere,
or a lofting of two curves, or a tangent developable, etc.,
give the (c,r,o) ring surface.



% ************************************************************

Machining and INTERFERENCE DETECTION emphasis

ringed \cap plane is important in analysis of sweep of object about polyhedral
environment
	it is a good problem because ringed \cap plane seems simple,
	BUT (1) circle \cap plane isn't trivial and
	    (2) combining the ringed \cap plane solutions for the entire
		polyhedral environment needs to be addressed
	It addresses the question: can you deal with a real problem using 
		the swept surface paradigm?

Graphics paper emphasizing ray casting.

The evangelism of the circle

Survey article on sweeping (Hopcroft's suggestion)

Alpha shapes (possibly for triangulation with holes; for clustering)

Cassini curves and tori (e.g., see Fischer and Lawrence)

% ************************************************************

design a solid modeler with ruled surfaces and cyclides (and planes and 
	quadrics)
	initially a toybox with intersection and design
	propose an addition to BRL or Parasolid

implementations

dependence on symbolic quartic solution is a weak foundation upon which to
	lay a theory, since the symbolic solutions are so convoluted

% *******************************************************************


d1^{2} + \alpha d2^{2} = k generalization of distance rep by two lines

	- k = 0 and \alpha < 0 is original alpha-bisector of two lines

	- can get sphere (lines through two antipodal points and 
			  diagonals of circumscribing cube that touches
			  sphere at those two points)
		  ellipsoid, elliptic cylinder, cone, hyp. of 1 and 2 sheets

	- definitely cannot get hyp cylinder, par cylinder, 
	  elliptic paraboloid

Point-line representation of quadrics

	- want classification table by sign of \alpha (plus angle of plane
	  and whatever else is necessary)

Arbitrary to normal-form reduction

Nearest neighbour: optimal representation of 
		   3d generalization of parabola 

	- in 2d with y=kx^{2}, the optimal m segments give an error
	  of \epsilon = \frac{k}{16m^{2}} (x_{m} - x_{0})^{3}
	     i.e., error reduces proportionate to 
		   square of number of segments

% ***********************************************

Voronoi diagram of lines (need intersection first)

	-application: safest-path motion planning
	-Voronoi diagram of curves is too tough at first
	 and not 3d anyway
	-involves hyperbolic paraboloids, which I know well

\title{Voronoi diagrams of lines in 3-space: closest and furthest VDs}

NEED FULL UNDERSTANDING OF INTERSECTION OF HYPERBOLIC PARABOLOIDS
(WHICH ARE THE BISECTORS OF THE LINES).
POSTPONE UNTIL INTERSECTION IS CONSIDERED.

We first establish that it is not a good idea 
to try to compute the Voronoi diagram of lines in 
3-space using Plucker coordinates and computing the Voronoi diagram
of the images of the lines in the Plucker space.
First of all, it is unclear whether distance is preserved by the Plucker
transformation: if it is not, then the Voronoi diagram of the image points
has no relationship to the desired Voronoi diagram of lines.
Even if distance is preserved, 
the Voronoi diagram in Plucker-space will yield a collection of lines
in 3-space, and the cells of the Voronoi diagram (the intersection points
of the lines etc.) still need to be computed.
Moreover, the collection of lines would be totally unstructured,
unlike the grouping into hyperbolic paraboloids if the Voronoi diagram
of lines is attacked directly in 3-space.

its combinatorial complexity 
point location in its cells

% **********************************************

Problems for C.-K.
------------------

Convex hull (motion planning)

	- how many segments in convex hull of curve?
			O(n^{2})?
			O(n)?

	- inflection points seem to be integrally related to the
	  convex hull of curves (although they do not represent
	  the exact vertices of the convex hull)

	- look at convex hulls of higher degree surfaces?
		(involves ruled surfaces)


Pencils
	
	- given two surfaces, what type of surfaces are
	  contained in their pencil (ideal)?

		e.g.: when do they contain a ruled surface?


Classification of cubics

	Get a good classification of cubic curves and surfaces.

		-Blythe has a `classification' of cubic surfaces
		 but it is not particularly good(?)

% ********************************************************************

key operations: curve-surface definition/description with points (input ops) *   ---
		intersection (Boolean operations)			     *	   -CREATION
		blending/fairing (smoothing operations)				   -
		rotation/translation (motion operations)		         ---

		sorting (ranking/order operation)			      ---
		convex decomposition, decomposition of high degree curve into	-CREATION
			a small number of low degree curves (decomposition 	-+ 
			into simpler components for easier treatment)	      ---ANALYSIS
			decomposition of algebraic surface into polyhedra
			and/or quadric surfaces
			convex hull algorithms are a start at this
			- the decomposition of a surface into 
			  convex components so that the theory of convex
			  bodies can be applied to each of them

		length/area/volume operations (analysis of solid)	      ---ANALYSIS
		hidden-surface elimination, silhouette (display operations)   ---

		parameterization/implicitization/inversion (indirectly)

curves of positive genus: elliptic functions          parameterization

------------------------------------------------------------------------

M.S theses

implementation of parameterization of ruled surfaces and space curves, other ops with 
	ruled surfaces
robot arm simulation (programming): 
bisector of algebraic curves / more complicated Voronoi diagrams (programming)
homotopy transformation: 
	my old work on homotopy (vs. potential) method for blending and fairing surfaces
	ruled surfaces (?)

------------------------------------------------------------------------

SIAM Journal on Computing
Algorithmica  (Hopcroft on editorial board, had entire issue dedicated to robotics)
Computer Vision, Graphics, and Image Processing (Sederberg et. al. on implicitization 
						 is here)
Computer Aided Design
Computer Aided Geometric Design  (Goldman et. al. on implicitization is here)
Journal of Algorithms
Discrete and Computational Geometry (tough according to Mike because they are glutted)
Discrete and Algebraic Geometry (the one s.t. Hopcroft is on editorial board)
IEEE Transactions on SMC

                          Journal Announcement
                          ********************

  Applicable Algebra in Engineering, Communication and Computer Science


Arrangements have been finalized with Springer-Verlag to publish a new journal
entitled "Applicable Algebra in Engineering, Communication and Computer Science"
The first issue of this quarterly journal will appear in the beginning of 1990.

AUTHORS ARE ENCOURAGED TO SUBMIT PAPERS. THIS IS A GOOD OPPORTUNITY FOR GOOD		**********
PAPERS TO BE QUICKLY PUBLISHED!

This journal aims to cover all aspects of applicable algebra in the fields of
Communication, Engineering, Artificial Intelligence and Computer Science. This
includes, but is not limited to, domains such as:
(i)   Engineering: Vision, robotics, system design, VLSI, signal processing,		*****
      fault tolerance and dependability of systems ....
(ii)  Communication: Signal theory, coding, error control techniques,
      cryptography, protocol specification ....
(iii) Computer science and Artificial intelligence: arithmetics, algorithm
      engineering, complexity, algebraic algorithms, computer algebra, logic		*****
      and functional programming, programming languages, term rewriting systems
      algebraic specification, theorem proving, graphics, modeling, knowledge		*****
      engineering, expert systems, artificial intelligence methodology ....

The scope of the journal is best illustrated by the two key-words in its title:
"Applicable Algebra". This implies that it does not cover domains which are not
linked to algebra such as applicable analysis or analytical methods to mention
two obvious ones.
Purely theoretical papers will not primarily be sought for even when they belong
to "algebraic" domains such as combinatorics, finite geometry, computer algebra
or cryptography for instance. But, papers dealing with problems in domains such
as commutative or non-commutative algebra, group theory, field theory, real or      *****
algebraic geometry for instance, which are of interest for applications in the
previously mentioned fields are relevant for this journal.
On the opposite, technology and know-how transfer papers from engineering which
either stimulate or illustrate research in Applicable Algebra do fit the scope
of this journal.

The editorial board consists of:
Th. Beth(Univ. of Karlsruhe), W. Buettner (Siemens, Munich),
J. Calmet (Univ. of Karlsruhe - Managing Editor), P. Camion (INRIA, Paris),
J. Cannon (Univ. of Sydney), J. Heintz (Inst. of Math., Buenos Aires),
C.M. Hoffmann (Purdue Univ.), H. Imai( Yokohama National Univ.),		   *****
D. Jungnickel (Justus-Liebig Univ. Giessen), E. Kaltofen (Rensselaer Poly-	   *****
technic Inst., Troy), D. Kapur (SUNY at Albany), P. Lescanne (CRIN, Nancy),
H. Lueneburg (Univ. of Kaiserslautern), H.F. Mattson (Syracuse Univ.),
T. Mora (Univ. of Genoa), J. Mundy ( GE Co., Schenectady),
H. Niederreiter (Austrian Academy of Sciences, Vienna), A. Poli (Univ. Sabatier,
Toulouse), S.A. Vanstone ( Univ. of Waterloo).

Submissions are to be sent to:

Calmet Jacques
e-mail : calmet@ira.uka.de  (csnet)      calmet@DKAUNI0I.BITNET (bitnet)
Universitaet Karslruhe - Institut fuer Algorithmen und Kognitive Systeme
Postfach 6980  -  D-7500 Karlsruhe 1  -  BRD  -   tel.: (49) 721 6084043
/* ---------- */


