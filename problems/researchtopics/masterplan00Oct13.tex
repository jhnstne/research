\documentclass[11pt,titlepage]{article}
\newif\ifFull
\Fullfalse
\input{header}
\newcommand{\add}[1]{#1}		% {{\bf #1}}
\newcommand{\plucker}{Pl\"{u}cker\ }

\DoubleSpace

\setlength{\oddsidemargin}{0pt}
\setlength{\topmargin}{.5in}	% should be 0pt for 1in
% \setlength{\headsep}{.5in}
\setlength{\textheight}{8.5in}
\setlength{\textwidth}{6.5in}
\setlength{\columnsep}{5mm}	% width of gutter between columns
% \markright{Master plan: \today \hfill}
% \pagestyle{myheadings}
% -----------------------------------------------------------------------------

\title{A research master plan}
\author{J.K. Johnstone\\UAB Geometric Modeling Lab}

\begin{document}

\maketitle

\section{Motion}

We have explored motion in several ways in our past work.
This includes the construction of bisectors of plane curves for safest path motion,
the construction of rational quaternion splines for motion in orientation space,
the construction of swept surfaces for interference detection,
and the study of ruled surfaces (motion of a line).
Our new work will address the following issues:
%
\begin{itemize}
\item   common tangents of plane curves
\item   tangents of a curve through a point (pole/polar/class)
\item   visibility graphs of plane curves
\item   penumbra/umbra classification of 2-space
\item   shortest path motion amongst plane curves
\item   common tangents on a sphere
\item   orientation planning amongst obstacles
\item   common tangents of surfaces
\item	lighting (umbra,penumbra) from smooth surface light in 3-space ('smooth shadows')
\item	finding a geodesic on a common tangent surface
\item	shortest path motion in curved 3-space environment
\item	convex hulls of curves and surfaces
\item	silhouette and visibility algorithms ('smooth silhouettes')
\item	striction curve of ruled surface; envelope of moving plane
\end{itemize}
%
Grant proposal themes:
\begin{itemize}
\item
Motion planning infrastructure in a smooth world
(visibility graphs, common tangent surfaces, visibility information)
\item orientation planning
\item corollary impact on lighting, sound, and shape
\end{itemize}
%
Grant sources (for further exploration):
\begin{itemize}
\item	ONR (McMullen, motion planning algorithms)
\item	Honda (orientation space motion planning amongst obstacles 
	and shortest path motion amongst smooth surfaces for 
	assembly-line robots, visibility analysis for car design, 
	human motion research?)
\item   Later
\begin{itemize}
\item   NSF (?, motion planning)
\item   NSF (?, shape analysis)
\item   ARO (Discrete Math and CS, shape analysis)
\end{itemize}
\item   After lighting research
\begin{itemize}
\item   NSF (?, lighting)
\item   Alias/Wavefront (lighting)
\end{itemize}
\end{itemize}

\begin{table}
\begin{tabular}{|c|c|c|}  	\hline
Topic & Conference & Due date \\ \hline
common tangents of plane curves & 	Shape Modeling & Oct. 15 \\ \hline
smooth visibility graphs & 	GI & Nov. 19 \\ \hline
pole/polar 		 & ACM SE Conference & Dec. 10 \\ \hline
smooth lighting &			Eurographics & Feb. 14\\ \hline
common tangent planes of surfaces &	Visualization & - \\ \hline
common tangents on S3 and shortest paths & ? & - \\ \hline
other uses of tangential curves	&	spring submission & - \\ \hline
convex hull &				Computational Geometry & - \\ \hline
shortest path motion &			R\&A 2002 & - \\ \hline
%
smooth shortest paths (2D won't cut it: need 3D) & ONR robotics & spring \\ \hline
? & Honda 		& Oct. 27 \\ \hline
modeling interest, MAYA &		Auburn architecture & fall \\ \hline
talk	& Illinois at Chicago & spring \\ \hline
\end{tabular}
\caption{Papers}
\end{table}



\clearpage


% mathematical elegance
% philosophical elegance
% religious elegance

Lines for visibility.
\begin{itemize}
\item	visibility--Canada--rococo
\item	visibility for viewing (can I see it?), lighting (can the light see it?),
		sound (can the violin see it?)
\item	just as with shape (Plato Meno), where the shape is the boundary of the
		object, the boundaries of visibility are of paramount 
		importance
\item	for sound, how many different ways can the sound bounce to me,
		and how many bounces does each path entail, and how far
		is each bounce?
		the cone of silence behind an obstacle;
		the cone of darkness behind an obstacle (the dark side of the moon);
		the cone of invisibility behind an obstacle.
\item	pole/polar, silhouette, common tangent, visibility graph in 2D;
		compute the visibility graph for a moving point in the plane
		(is there an analog in space?);
		compute the moving silhouette as the eye travels about a room
		
\item	lighting needs the global lit (or dark) locus, while motion needs
	a specific path along the boundary of this locus;  in this sense,
	motion is a second-order use of visibility information,
	while lighting is a first-order use.  However, motion is a less 
	realistic application of visibility since following the true shortest
	path is a wee bit forced, while light and sound truly do follow
	the shortest (visible) path naturally.
\item	
	all of these visibility problems have been studied, but in the polygonal
	discrete realm, not the continuous smooth one.
	
\item	pole/polar of a Bezier curve
\end{itemize}
		












\clearpage
	
\section{Motion}

\begin{enumerate}
\item Position
\begin{enumerate}
\item Rigid robot
\begin{enumerate}
\item Shortest path
\begin{enumerate}
\item In the plane % ({\em common tangent space})
\item In 3-space
\item On a 2-surface
\end{enumerate}
\item Safest path
\begin{enumerate}
\item In the plane % ({\em bisector})
\item In 3-space
\item On a 2-surface
\end{enumerate}
\end{enumerate}
\item Linked robot
\begin{enumerate}
\item Amongst obstacles
\begin{enumerate}
\item configuration space obstacle
\end{enumerate}
\end{enumerate}
\end{enumerate}

\item Orientation
\begin{enumerate}
\item In 3-space 
\begin{enumerate}
\item Free space % ({\em quaternion spline})
\item Amongst obstacles
\end{enumerate}
\end{enumerate}

\item Workspace determination
\begin{enumerate}
\item Swept surfaces % ({\em shape-preserving swept surfaces}, {\em ruled surfaces})
\item Intersection algorithms % ({\em ringed/cyclide intersection})
\end{enumerate}

\item Morphing (shape under motion: dynamic shape)
\begin{enumerate}
\item Traditional morphing
\item Contour reconstruction as morphing % ({\em smooth contour reconstruction})
\item Scattered data reconstruction as contour reconstruction
	(hence as morphing)
\end{enumerate}
\end{enumerate}


\clearpage

\section{Shortest path motion of a rigid robot in the plane, position only}

\begin{enumerate}
\item Of a point among polygonal obstacles
\begin{enumerate}
\item visibility graph (Lozano-Perez)
\end{enumerate}
\item Of a point among curved obstacles
\begin{enumerate}
\item tangent space
\begin{enumerate}
\item common tangent of two curves
\item polar of a point and a curve
\end{enumerate}
\end{enumerate}
\item Of a polygon among polygonal obstacles
\begin{enumerate}
\item Minkowski sum (Lozano-Perez)
\end{enumerate}
\item Of a curve among curved obstacles
\begin{enumerate}
\item envelope
\end{enumerate}
\end{enumerate}

{\bf Value added: motion in a curved world.}
Motion algorithms are improved through the use of curved, rather than polygonal, obstacles.
Motion in a curved world is more efficient, since visibility graphs are smaller.
It is also more accurate.
Using the same theory, we also have improved lighting in a curved world.

\vspace{1in}
Fun applets
\begin{itemize}
\item	Pole/polar of a circle
\item   Inversion in a circle
\end{itemize}

Motion in 3-space amongst polyhedra: Schwartz/Sharir and Reif/Storer
(Shortest paths in Euclidean spaces with polyhedral obstacles, 1985 Brandeis
Tech Report).

\clearpage

\section{Orientation control of a rigid robot in 3-space}

{\bf Value added: improved rational quaternion splines.}
Efficient rational quaternion splines can improve all aspects of animation that use
quaternions, such as motion editing, motion optimization, physics-based animation,
and high-level animation.

Fun applets
\begin{itemize}
\item Quaternion associated with a cube's orientation.
\end{itemize}
	
\clearpage

\section{Workspace determination}

\begin{enumerate}
\item Workspace determination
\begin{enumerate}
\item Swept surfaces
\begin{enumerate}
\item Pythagorean hodographs for rational perfect sweeps
\end{enumerate}
\item Surface intersection
\begin{enumerate}
\item Ruled surface intersection (Heo, CAD, 1999; Heo thesis, 2000)
\item Ringed surface intersection (Heo 2000)
\end{enumerate}
\end{enumerate}
\end{enumerate}

Fun applets
\begin{itemize}
\item Intersection of two lines
\end{itemize}

Add ruled surface to modeling vocabulary (representation, intersection,
	ray casting)
	
Sources of ruled surfaces: common tangents between 2 surfaces,
	cylinders, ruled quadrics, envelope of link, part of convex hull
	of collection of surfaces

\clearpage

\section{Morphing: Dynamic shape}

Dynamic shape is shape defining a motion rather than a static boundary.
It can also refer to a shape that changes over time, either morphing between two shapes
or vacillating in some other way.

\begin{enumerate}
\item Dynamic shape
\begin{enumerate}
\item The locus of the reference point of a moving object
\item Morphing
\begin{enumerate}
\item Traditional morphing
\begin{enumerate}
\item Sederberg
\item Parent
\end{enumerate}
\item {\bf Contour reconstruction as morphing}
\begin{enumerate}
\item Contour reconstruction as the spacetime envelope or locus of a morph
\item What does morphing have to say to contour reconstruction?
	Do morphing algorithms generate good contour reconstructions?
\item What does contour reconstruction have to say about morphing?
	Can contour reconstruction techniques improve morphing?
\end{enumerate}
\item Scattered data reconstruction as contour reconstruction
\begin{enumerate}
\item {\bf Extraction of contours from scattered data}
\end{enumerate}
\end{enumerate}
\end{enumerate}
\end{enumerate}

{\bf Value added: Insight into morphing and reconstruction, 
new algorithms for scattered data and contour reconstruction,
and a unification of morphing with reconstruction.}

Fun applets
\begin{itemize}
\item Morphing between two input polygons using minimal area triangulation.
\end{itemize}

\clearpage

\section{Contour reconstruction}

\begin{enumerate}
\item Clean up data.
\begin{itemize}
\item remove duplicates and self-intersections
\item impose consistent ccw orientation
\item output of 'recon -j file.rawctr'
\item we want to resample the curves implied by the contour data to clean it 
	up; this may be appropriate at this stage [at present, we postpone this
	resampling, since we don't use it at the moment until after the 
	topology is computed; but we could use the original data points
	in the topology computation even if the resampling already existed
	and was already considered dominant]
\end{itemize}
\item Decompose object into metatubes, canyons and canyon caps.
\begin{enumerate}
\item Begin to locate metatubes, canyons and canyon caps, by locating branches
\begin{itemize}
\item reconstruct topology
\item which contours connect to which contours, especially at branches?
\item methodology: MST of contour graph
\item output of 'recon file.rawctr'; branches stored in file.topoctr
\end{itemize}
\item Refine location of canyons and caps
\begin{itemize}
\item mediate between tubes at branches [build canyon bridges]
\item Purpose: to add a tube at branches merging between parent stem and
	offspring branches (picture of merging tube between ordinary tubes)
\item Second purpose: to define the canyons at branches
\item build bridgeheads to stitch branching contours together one section longer
	for a smooth transition
\item the bridgeheads define the canyon at a branch (the bridges and 
	contour segments between the bridgeheads are the boundary of the canyon)
\item methodology: curve match with parent contour
\item output of 'recon file.topoctr'; bridgeheads stored in file.bridgectr
\item bridges need to be temporarily defined
\begin{itemize}
\item the bridges define the canyon in the contour plane
	(but notice that the bridge will not lie on the final surface:
		it will be dropped down below the contour plane first)
\item the bridges also indirectly define the boundaries
      of the canyon caps, as follows: after resampling and triangulation of the
	supercontour with the parent contour, the points
	of the parent contour connected to a bridge by the triangulation
	define one of the boundaries of the canyon cap, the bridge defines
	another boundary, and the obvious connections between them define
	the other two boundaries.
\item after the bridges have been used for the canyon quadrangulation
	and the definition of the canyon caps, they are dropped to 
	a plane oblique to the contour plane, using tangent ribbons
	at the bridge endpoints to define a Hermite curve.
\end{itemize}
\end{itemize}
\item Decompose surface into metatubes, canyons, and canyon caps
\begin{itemize}
\item Data structure for canyons, caps, metatubes?
\item Canyon = 2 contour curve segments; data points on these curves
\end{itemize}
\end{enumerate}
\item Define object polygonally
\begin{enumerate}
\item define metatubes, polygonally 
\begin{itemize}
\item triangulate tubes (ignoring canyons)
\end{itemize}
\item define caps, polygonally
\begin{itemize}
\item identify subset of triangulation that defines cap
\end{itemize}
\item define canyons, polygonally 
\begin{itemize}
\item quadrangulate canyons
\item could sample canyon at some denser/uniform rate for quadrangulation vertices or
	could use original data points; resampling is favoured for 
	allowing more optimal quadrangulations (and allowing a pure quadrangulation
	rather than a quadrangulation/triangulation), but discouraged for
	possibly generating more patches
\item use the existing resampling as the foundation of the vertices,
	but {\bf allow vertices to be skipped} as you build the quadrangulation,
	so as {\bf to favour optimal rectangles} 'directly across' the canyon (not slanted);
	that is, quadrangulate a subset of the resampled data points
\item only use the active (nonskipped) data points to calculate tangent ribbons
	below, so that ribbons on the side of a canyon quadrangle are only
	one segment long (during Coons patch construction)
\item Note: could triangulate and later use triangular Coons patches,
	but this is a topic for future study (M.S.)
\item could even represent canyon as 2n curve segments and quadrangulate
	by generating quadrangles at arbitrary points of the curves
	(effectively resampling guided by the quadrangulation);
	the points of this quadrangulation would then become the seeds
	of the flowlines for the metatubes;
	this is the most elegant solution since it does not depend on any
	sampling; the canyon is simply a collection of curves
\end{itemize}
\end{enumerate}
\item Define object's curve network.
\begin{enumerate}
\item define metatube's curve network
\item define canyon's curve network 
\begin{itemize}
\item build curves along interior edges of quadrangulation
\item basic idea is Hermite curves, but difficult since tangents
	at two ends of edge are not coplanar and may be quite wildly different,
	leading to an awkward Hermite curve; a compromise is desirable,
	but one that allows a good Coons patch to be constructed later;
	this shouldn't be too hard since the shape of this boundary curve
	of the patch is a free parameter in the Coons patch
	and there is no need to follow the tangents at the endpoints
\end{itemize}
\item define cap's curve network
\end{enumerate}
\item Define C1 boundary information between components
\begin{enumerate}
\item preprocessing: seed flowlines from each data point of boundary contours
	of metatubes (only using nonskipped data points on canyon)
\begin{itemize}
\item   note that the data point sampling rate has an impact here
\item	this inherently includes canyons and covers the necessary points
	for later ribbon definition 
\item   ensure that all data points of shared boundary are the same,
	with respect to both metatubes on either side
\end{itemize}
\item define metatube's C1 boundary information
\begin{itemize}
\item	build tangent ribbons at boundaries of metatube, using flowlines
\item	Want boundary ribbons of neighbouring metatubes to match:
	how shall they be specified?
	1) Automatically from patch, but this requires an ordering of the
		metatubes so that one ribbon can be inherited to the next
		for $C^1$ continuity; we do not want a different surface
		depending on whether we traverse the metatubes up or down.
	2) From triangulation: at each data point, flow up to next contour
		and flow down to previous contour; using these three points,
		define a parabola and read off its tangent
		(at object boundaries including canyons, just use two points 
		to define the tangent directly, or perhaps to define a parabola
		as in Bessel end tangents for a less acute tangent).
		Then interpolate these tangents across the entire curve
		(using the same approach as interpolating points).
		Note to only use the nonskipped data points from the canyon.
\end{itemize}
\item define canyon's C1 boundary information
\begin{itemize}
\item for missing ribbons of canyon rectangles,
	artificially and arbitrarily construct as orthogonal to Hermite boundary
\end{itemize}
\item define cap's C1 boundary information
\end{enumerate}
\item Define object's surface.
\begin{enumerate}
\item define metatube's surface
\begin{itemize}
\item tensor product Bezier patches guided by triangulation
\end{itemize}
\item define canyon's surface [Coons patches]
\begin{itemize}
\item draw Coons patch by evaluating u,v points at a dense sampling on the patch
	and drawing resulting triangulation (not too different from 
	display of Bezier surface)
\item could use triangular Coons patch (BBG or Nielson)
\end{itemize}
\item define cap's surface [Coons patches]
\end{enumerate}
\end{enumerate}

Get DeRose/Sloan triangular surface patch smooth solution from triangulation
software up and running for a comparison.

16 vs. 32-bit images.

Fun applets
\begin{itemize}
\item Supercontour construction
\item Quadrangulation
\item Identification of canyon cap.
\item Flowline construction.
\item Coons patch construction
\end{itemize}

\begin{enumerate}
\item Reconstruction from morphing algorithm
\item Reconstruction from transaxial and coronal (or sagittal) contours
\end{enumerate}

{\bf Value added: a smooth reconstruction with a natural definition of
	canyons, with potential insight into morphing (of which a polygonal
	reconstruction is less capable, since the canyon is either artificially
	dropped down halfway or fully dropped down, so the natural
	morphing of two contours into one, which is defined by the shape
	of the canyon, cannot be elegantly done in a polygonal reconstruction).
	
	Insight into the 'correct' shape between two curves (or n brother curves
	and a parent curve).
	
	Insight into shape.
}

\vspace{.5in}

Applications of reconstruction
\begin{itemize}
\item	Shape acquisition for animation, virtual reality worlds, 
	engineering, medicine.
\item	Anatomical modeling; tissue modeling (adrenal,ganglion)
\item   Advantages over Marching Cubes and scattered data reconstruction
	(why not use marching cubes for scattered data, if it is so great?)
\end{itemize}

\clearpage

\vspace{-.5in}

\section{The large picture}

We are interested in motion, its description, its planning.
%
\begin{verbatim}
visibility for shortest path motion	 (also applicable to lighting)
  visibility graph amongst curved obstacles
	2-space: common tangents, tangents from a point
		tangent space, polars, dual curve (normal space, silhouette)
	 		lines
				Plucker space
	3-space: bipolar curves of common tangency
		 common tangent ruled surfaces
		 geodesics on surfaces (contact motion + shortest path between 2 solids)
				
motion in orientation space
	quaternion splines
		rational quaternion splines (GI)
	motion amongst orientation obstacles
		tangent space on a nonplanar Riemann surface
			common tangents on S3
	pure orientation and orientation + translation
	motion editing
	animation
				
kinematics (rather than point robot)
	line geometry
workspace determination
	envelopes, swept surfaces
		sweeping an object while preserving its shape -> swept surface (Eurographics)
	intersection algorithms
		ringed/cyclide intersection (CAGD)
safest path motion
	bisectors (CAGD)
\end{verbatim}

\end{document}
