\documentstyle[times]{article} 
\makeatletter
\def\@maketitle{\newpage
 \null
 \vskip 2em                   % Vertical space above title.
 \begin{center}
       {\Large\bf \@title \par}  % Title set in \Large size. 
       \vskip .5em               % Vertical space after title.
       {\lineskip .5em           %  each author set in a tabular environment
        \begin{tabular}[t]{c}\@author 
        \end{tabular}\par}                   
  \end{center}
 \par
 \vskip .5em}                 % Vertical space after author
\makeatother

\parskip=.2in
\parindent=0pt
\setlength{\oddsidemargin}{0pt}
\setlength{\topmargin}{-.25in}	% technically should be 0pt for 1in
\setlength{\headsep}{0pt}
\setlength{\textheight}{8.75in}
\setlength{\textwidth}{6.5in}
\setlength{\columnsep}{5mm}		% width of gutter between columns

\begin{document}

\centerline{Research plan for the academic year '96-97: John K. Johnstone}

\section{Journal papers and SIGGRAPH}

The most significant step towards a successful funding program
is the maintenance of a productive research effort.
Journal papers constitute the permanent record of this effort,
are a major component of the review criteria for grant proposals, 
and advertise one's work to the very people who will be making decisions 
on funding.
I will be concentrating on the submission to prestigious journals,
such as Computer Aided Geometric Design, ACM Transactions on Graphics,
and IEEE Computer Graphics and Applications, of several
recent results on curve design, animation, swept surfaces,
contour reconstruction, and biomedical visualization, all of which
have already appeared in conference papers.

Arguably the most important, and certainly the most visible, forum
for graphics research is SIGGRAPH,
the annual conference in the U.S..  Submission of papers to this conference
require a great deal of systems work and video preparation.
A major focus of my fall quarter will be the preparation of a paper
for submission to SIGGRAPH '97.

\section{National Science Foundation}

I will submit to the call for proposals in
`optimization, numeric computing, and computer graphics' within the subdivision
of Numeric, Symbolic and Geometric Computing (Dr. Kamal Abdali, Director)
in the Division of Computer and Computation Research of the NSF.
This has a deadline of September 15, 1996.
The topic of my proposal will be anatomical modeling.
The rearrangement of Dr. Abdali's subdivision to include computer graphics
should allow a more professional evaluation.

\section{Modeling of the knee}

I have a pending proposal to the NIH with Dr. Alan Eberhardt of
Mechanical Engineering and Dr. John Cuckler of Orthopedic Surgery.
I will continue this project, which involves the modeling of the knee
for stress evaluation and surgical planning.
We are developing a mechanism to evaluate knees from MRI within one day,
which involves contour segmentation, geometric modeling, finite element
mesh design, and visualization.
I hope to publish some of this work over the next year.

\section{Modeling and visualization of electrical activity across the heart}

I have a pending proposal with Dr. Gregory Walcott to Elecath Incorporated,
which has received promises of funding, and I will continue this work
during the next year.
The Cardiac Rhythm Measurement Lab, of which Dr. Walcott is a member, 
is an excellent research lab and I value my continued collaboration
with this group.
This project involves the modeling of cathethers within the ventricle
and the ventricles themselves from fluoroscopy,
perhaps combined with a general MRI model, and the calculation and subsequent
visualization of electrical activity across the ventricle from electrodes
on the cathether.

\section{Infomedia}

Infomedia Inc. is presently supporting one of our graduate students
in a research assistantship position, and I hope that this relationship 
can continue.  Infomedia is a webpage design group.

\section{Projects to be explored}

There are a number of potential projects that I will explore during
the upcoming year.
I will discuss the possibilities of a project on fluid flow
through anatomical models with a group in Huntsville and Arizona State University.
A colleague that I met in Germany at the Dagstuhl geometric modeling workshop
has a relationship with a company in Huntsville, which contains ex-students,
in research on fluid flow.
I have no expertise on fluid flow, but am developing methods for
the reconstruction of branching structures such as arteries and veins
that could create anatomical models to be used in this fluid flow.

A course in a subject related to graphics, such as OpenGL, Open Inventor or VRML,
could be offered in the department's career enhancement program.
I will explore the market and review materials for such a course.
In order to teach a course on any of these topics, 
the department would need to purchase
a license for OpenGL and Open Inventor on the Suns.

I continue to look for opportunities in the Department of Defense.
ONR (Office of Naval Research)
has no programs in computer graphics or geometric modeling,
except in `Volume Visualization' (not one of my research areas) 
under Larry Rosenblum.
AFOSR (Air Force Office of Scientific Research) has a
program in `Virtual Environments for Training' which includes work
in solid modeling as one of its many components, 
but these are very large awards for large groups,
1-3 awards of \$1.5-4 million per year for 5 years,
which would require expertise in many other areas in which our department
(small as it is) does not perform research.
The Army Research Office has a division of Discrete Mathematics and
Computer Science (Dr. Ming Lin, Director)
that contains solid modeling and visualization under its umbrella.

I hope to give talks at other universities during the next year,
especially at the University of North Carolina at Chapel Hill,
where I have been invited.
This talk might be combined with a visit to the Army Research Office,
which is also in the Research Triangle.

\end{document}
