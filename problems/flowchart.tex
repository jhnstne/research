\documentclass[11pt]{article}

\newif\ifFull
\Fullfalse

\makeatletter
\def\@maketitle{\newpage
 \null
 %\vskip 2em                   % Vertical space above title.
 \begin{center}
       {\Large\bf \@title \par}  % Title set in \Large size. 
       \vskip .5em               % Vertical space after title.
       {\lineskip .5em           %  each author set in a tabular environment
        \begin{tabular}[t]{c}\@author 
        \end{tabular}\par}                   
  \end{center}
 \par
 \vskip .5em}                 % Vertical space after author
\makeatother

% non-indented paragraphs with xtra space
% set the indentation to 0, and increase the paragraph spacing:
\parskip=8pt plus1pt                             
\parindent=0pt
% default values are 
% \parskip=0pt plus1pt
% \parindent=20pt
% for plain tex.

\newenvironment{summary}[1]{\if@twocolumn
\section*{#1} \else
\begin{center}
{\bf #1\vspace{-.5em}\vspace{0pt}} 
\end{center}
\quotation
\fi}{\if@twocolumn\else\endquotation\fi}

\renewenvironment{abstract}{\begin{summary}{Abstract}}{\end{summary}}

\newcommand{\SingleSpace}{\edef\baselinestretch{0.9}\Large\normalsize}
\newcommand{\DoubleSpace}{\edef\baselinestretch{1.4}\Large\normalsize}
\newcommand{\Comment}[1]{\relax}  % makes a "comment" (not expanded)
\newcommand{\Heading}[1]{\par\noindent{\bf#1}\nobreak}
\newcommand{\Tail}[1]{\nobreak\par\noindent{\bf#1}}
\newcommand{\QED}{\vrule height 1.4ex width 1.0ex depth -.1ex\ } % square box
\newcommand{\arc}[1]{\mbox{$\stackrel{\frown}{#1}$}}
\newcommand{\lyne}[1]{\mbox{$\stackrel{\leftrightarrow}{#1}$}}
\newcommand{\ray}[1]{\mbox{$\vec{#1}$}}          
\newcommand{\seg}[1]{\mbox{$\overline{#1}$}}
\newcommand{\tab}{\hspace*{.2in}}
\newcommand{\se}{\mbox{$_{\epsilon}$}}  % subscript epsilon
\newcommand{\ie}{\mbox{i.e.}}
\newcommand{\eg}{\mbox{e.\ g.\ }}
\newcommand{\figg}[3]{\begin{figure}[htbp]\vspace{#3}\caption{#2}\label{#1}\end{figure}}
\newcommand{\be}{\begin{equation}}
\newcommand{\ee}{\end{equation}}
\newcommand{\prf}{\noindent{{\bf Proof} :\ }}
\newcommand{\choice}[2]{\left( \begin{array}{c} \mbox{\footnotesize{$#1$}} \\ \mbox{\footnotesize{$#2$}} \end{array} \right)}      
\newcommand{\ddt}{\frac{\partial}{\partial t}}
\newcommand{\ovalbox}[1]{\begin{picture}(70,30)
	\put(0,-12){\makebox(70,30){#1}} \put(35,3) {\oval(70,30)} \end{picture}}
\newcommand{\bigovalbox}[1]{\begin{picture}(100,30)
	\put(0,-12){\makebox(100,30){#1}} \put(50,3) {\oval(100,30)} \end{picture}}
\newcommand{\ra}{$\rightarrow$}
\newcommand{\lra}{$\longrightarrow$}

\newtheorem{rmk}{Remark}[section]
\newtheorem{example}{Example}[section]
\newtheorem{conjecture}{Conjecture}[section]
\newtheorem{claim}{Claim}[section]
\newtheorem{notation}{Notation}[section]
\newtheorem{lemma}{Lemma}[section]
\newtheorem{theorem}{Theorem}[section]
\newtheorem{corollary}{Corollary}[section]
\newtheorem{defn2}{Definition}

% \ifFull                                          
\SingleSpace
% \else
% \DoubleSpace
% \fi

\setlength{\oddsidemargin}{0in}
\setlength{\evensidemargin}{0in}
\setlength{\headsep}{0pt}
%\setlength{\topmargin}{0pt}
%\setlength{\textheight}{8.75in}
\setlength{\topmargin}{-0.4in}
\setlength{\textheight}{9.00in}
\setlength{\textwidth}{6.5in}
\setlength{\headsep}{.2in}

% ****************************************************************************

\title{}
\author{John K. Johnstone\\
	Dept. of Computer and Information Sciences\\
        The University of Alabama at Birmingham\\
        125 Campbell Hall, 1300 University Boulevard\\
        Birmingham, Alabama  35294-1170 USA.\\
	johnstone@cis.uab.edu\\
	http://www.cis.uab.edu/}

\begin{document}

(direct) acquisition of contour data:

\fbox{object} \ra \ \bigovalbox{imaging/sectioning}
\ra \ \fbox{planar images} \ra \ \bigovalbox{segmentation} \ra \ 
\fbox{contours} 

\vspace{.2in}

(indirect) acquisition of contour data:

\fbox{object} \ra \ \ovalbox{scanning} \ra \ \fbox{data points} \ra \ 
\ovalbox{structuring} \ra \ \fbox{contours}

\vspace{1in}

{\bf Theme:} treat contours {\em always} as curves,
	{\em never} as points or polygons.

{\bf Challenge:}
In segmentation, interactively create curves through sampled points.
In correspondence, compute approximate centroids and junction points
directly from curves.
For example, centroids from a coarse sampling of curve and
junction points from closest points or common tangents of curve.

\vspace{1in}

\fbox{contours} \ra \ \bigovalbox{correspondence} \ra \ \fbox{contours with topology} 

\fbox{contours} \ra \ \bigovalbox{triangulation} \ra \ \fbox{contours and triangulation} \\
\hspace*{.6in} \ra \ \ovalbox{flow} \ra \ \fbox{isoparametric curves} 
\ra \ \fbox{isogrids}
\ra \ \fbox{metatube patches} \\
\hspace*{.6in} \ra \ \bigovalbox{canyon filling}
\ra \ \fbox{metatube and canyon patches}

\clearpage

MRI = `magnetic resonance imaging' scan \\
NMRL = UAB Nuclear Magnetic Resonance Lab \\
VH = Visible Human Project data (and secondary sources) \\
Bar = Gil Barequet's contour database \\
LW = LiveWire segmentation algorithm \\
i-MEST = intelligently weighted MEST algorithm \\
j-pt = junction point \\
Cy = Cyberware laser scan \\
Wash = University of Washington repository \\
Stan = Stanford University repository \\
resampling = may or may not preserve original points; for good flowlines \\
* = resample at this stage to use existing Sloan data sets \\
source of planar images: MRI images from NMRL and VH \\
source of contours: NMRL, VH, Bar \\
correspondence uses i-MEST and j-pt identification \\
contours after correspondence: evenly and densely sampled; scaled to unit cube;\\
	ALIAS info for branching; ADJPREV and ADJNEXT for branching and
	identification of junction points (canyons);
	rotated so that each contour starts with a junction point (if one exists) \\
contours after triangulation: both merged supercontours and original contours
FKU = Fuchs-Kedem-Uselton minimal area triangulation \\
ACF = Augmented Contours Format

\clearpage

\fbox{biological specimen} $\stackrel{\mbox{microscopy}}{\longrightarrow}$ 
\fbox{microscopic sections} $\stackrel{\mbox{segmentation}}{\longrightarrow}$
\fbox{contours} $\ldots$

\vspace{1in}

\begin{tabular}{ccccc}
\fbox{star-shaped visible object} 
& $\stackrel{\mbox{Cyberware}}{\longrightarrow}$ 
& \fbox{scattered data} 
& $\stackrel{\mbox{trivial organization}}{\longrightarrow}$
& \fbox{nonbranching contours}
\\
& & $\uparrow$
\\
& & Cy \\
& & Wash \\
& & Stan \\
\end{tabular}

\end{document}
