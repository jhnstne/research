\documentclass[10pt]{article}
\newcommand{\DoubleSpace}{\edef\baselinestretch{1.4}\Large\normalsize}
\DoubleSpace

% set the indentation to 0, and increase the paragraph spacing:
\parskip=15pt plus1pt
\parindent=0pt
% default values are \parskip=0pt plus1pt \\ \parindent=20pt for plain tex.
\setlength{\oddsidemargin}{-.8in}
\setlength{\evensidemargin}{0pt}
\setlength{\headsep}{-.3in}
\setlength{\topmargin}{0pt}
\setlength{\textheight}{10in}
\setlength{\textwidth}{6.5in}

\pagestyle{empty}

\begin{document}

\begin{tabular}{llll} \hline
Student & Project & Papers being read & Papers being written \\
Lin Yang        & canonical views (image mining) & Raguram, Yang & Level 2 proposal \\
David O'Gwynn   & mesh segmentation & BFS/Fiedler & curvature, Level 2 proposal\\
--              & registration & ICP/features \\
--              & shape descriptors & curvature, suggestive contours \\
--              & shape software \\
--              & parsing of shape data (ctr08, obj, B-spline) \\
Sagar Thapaliya & obstacle-avoiding fitting & Farin, shortest/safest path  & \\
Patrick Butler  & terrain reconstruction & FKU, Barequet & USGS parsing \\
{\bf Gilbert Weinstein} & differential geometry of shape models \\
Eric Frees      & \\
Douglas Ross    & finding pockets & bitangency, feature-based reg 
                & liquid vessels, can you handle it?\\
Mark Dillavou   & {\tiny inserting 3d models from CT into surgical field} 
                & {\tiny Z. Wood mesh extraction from volumes} & \\ \hline
\end{tabular}

\begin{tabular}{lllllll} \hline
Project & Topic of paper & Student & Submission \\
Shape at a branch & Cassini/rosette branching & & Eurographics\\ % ****
                  % called intermediate contour approach
Shape at a branch & pincered bridges & & WSCG \\
Shape at a branch & m-to-n 'divide and conquer to 2-to-1' algorithm & & NSF \\
Shape at a branch & determining topology of a contour dataset \\
Shape at a branch & definition (and parsing) of ctr08 format \\
Shape at a branch & smooth lofted intermediate contour reconstruction & & \\ % contour/paper/2007 (e.g., ctr08text.tex) 
Shape software suite & \\
Interpolation & obstacle-avoiding interpolation & & tech report, CAD 10 \\       % *
Mesh semantics & Fiedler for mesh segmentation w. weakening              % *
               & David & Meshing Roundtable \\ % rEGISTRATION/src
Image mining & canonical views from photo collections & Lin & Google \\ 
Mesh semantics & feature detection: characterization of liquid vessels & Douglas \\
Fitting & tutorial on cubic B-spline/Bezier fitting & Sagar \\
Terrain & cleanup/topology-extraction from topographical maps & Patrick & Google \\ % *
Eric project & registration; architectural geometry; symmetry; & Eric & NSF UG research \\
             & lines on meshes; Algorithmic Beauty of Plants;  \\
Registration & registration of two contour meshes & David, Doug, Eric, Marcus & MICCAI \\    % *
Anatomical modeling & \\
Ruled surface & finding directrix curve \\
Tangent space & tangential curve/surface; bitangency \\
Tangent space & tritangency, pockets, protrusions & & CAD \\
Tangent space & visibility graph \\
Camera control& rational quaternion splines \\ \hline
\end{tabular}

\begin{tabular}{lll} \hline
Funding source & Project & Due Date \\
Google (research.google.com) & Shape software suite; image mining & 8.15 \\
NSF IIS/Human-Centered Computing (Computer Graphics) & Image mining & 12.1\\
NSF CCF/Algorithmic Foundations (Computational Geometry) & Shape at a branch & 12.1\\
NIH & Reconstruction for quantitative analysis \\ \hline
% Insertion of CT models into surgical suite \\ \hline
\end{tabular}

\clearpage

{\bf Google abstract}:

1) interpolation as learning:
interpolation of point data (e.g., multidimensional learning);
interpolation is the generation of complete data from partial data: 
this is a form of learning

2) interpolation from contour datasets (e.g., topographical maps)

3) construction of natural camera paths interpolating along a road in Google Maps

{\bf NSF abstract}:

1) a software suite for shape modeling including meshes, splines, contours

2) the definition of ramiform shape (shape at a branch);\\
in particular, building ramiform shape from a contour fragment (e.g,. 1-to-2 fragment)

{\bf Munden abstract}:

registration between two slightly different meshes using feature analysis

A challenge for the quantitative assessment of therapeutic responses is the ability 
to register (align) datasets across imaging modalities, imaging platforms in different 
hospitals, and across a time series. 
The ability to compare and integrate scans from several imaging modalities and platforms 
involves rigid registration, 
while non-rigid registration (or perhaps partial registration of stable subareas) 
is particularly useful for analysis across a time series, 
where the geometry will gradually change. 
Many powerful techniques for registration have been developed in computer vision 
for the registration of stereo image pairs and, particularly in recent years, in 
computer graphics for the registration of point data from multiple scans of an object. 
Many of these techniques rely on variants of ICP or iterated closest 
point (Besl, 1992; Rusinkiewicz, 2008), which minimizes least-squares error between 
two point sets, and include extensions to non-rigid registration (Brown, July 2007). 
Recently, a powerful method (4PCS) has been developed for registration of two 
noisy datasets with minimal overlap, which is efficient and robust to 
outliers (Aiger, August 2008). 
ICP is well adapted to rigid registration across platforms and modalities, and 
some forms of non-rigid registration across a time series 
through its extension (Brown, July 2007),
while 4PCS is promising for the registration of fixed subareas 
within a globally changing dataset across a time series. 
Distinctive features such as ridges (Pennec, 2000) will also be used to guide the registration,
leveraging work on the detection of feature lines on meshes (Rusinkiewicz, 2008).
Comparison of two patients, or of a patient to a digital atlas, 
requires more radical deformation of structure and therefore different techniques, 
such as (Eckstein, 2007) , which could also be explored.
Successful development of this project would allow harmonization 
of data collected from variable platforms to facilitate accurate analysis 
of tumor response and validation of imaging biomarkers.

\clearpage

{\bf Shape formats}: points (e.g., scan), 
               {\bf mesh} (data structure = winged/corner/quadedge, 
                     software = Rusienkiwicz/Heckbert/Joy), 
               architectural, 
               {\bf Bezier/B-spline}, 
               implicit (eg, piecewise algebraic), particles,
               *contour, range map, voxels, 
               {\bf keyframes} (positions/orientations)


{\bf Interpolation is the addition of structure to a shape, using hints and deduction from
the existing data.  Interpolation as learning.}

{\bf Shape interpolation} (translation/transformation): 
              point to mesh (e.g., Amenta, Dey); 
              mesh to smooth model (e.g., Mann, Peters, Qin); 
              contour to mesh (FKU, Sederberg, Barequet);
              hole filling (Hahn); 
              contour to smooth surface (Gabrielides);
              structure from motion (images -> camera path)

Collect points and contours from a known mathematical algebraic surface,
a known parametric surface, and a known mesh.
This will allow a quantitative assessment of the quality of a reconstruction
based on the ground truth of the original shape model.

Invite Stephen Mann, Jorg Peters, and Hong Qin to talk about mesh to smooth model.
Invite Tom Sederberg, Gill Barequet, and Gabrielides to talk about contour reconstruction.
Invite Nina Amenta to talk about mesh reconstruction.
Invite Gerald Farin to talk about curve and surface interpolation.

GRAIL desks: David, Lin, Sagar; potential: Eric/Patrick share the 4th desk;
                    members but no space necessary: Doug/Mark.
Logo brainstorm.  
Tech report brainstorm (call the tech series 'patches').  
Paper submissions over the next 6 months.
Monthly jam session on research progress.
Monthly party (skating/movie/music).

\begin{tabular}{ll} \hline
Course & Textbook \\
% CS201 & {\em Sedgewick} and Wayne \\
CS250 & {\em Scheinerman} \\
CS350 & Hopcroft, Motwani and Ullman; Prusinkiewicz and Lindenmayer; 
        dragon book (lexical analysis; parser) \\
% CS753 & de Berg; CGAL \\
% CS770 & Shirley; Shreiner (OpenGL) \\
CS771 & Farin; consider Goldman; Knotty \\
CS772 & Walker (alg curves); Thorpe (diff geometry); Singer and Thorpe (topology); Warren \\
CS780 & Golub and van Loan; Demmel; Trefethen and Bau; (Halmos, Lang); Anderson (LAPACK), Higham (MATLAB) \\
% CS792 & Hartley and Zisserman \\ 
\hline
\end{tabular}

\begin{tabular}{ll} \hline
Tool & Resource \\
emacs & GNU Emacs Tutorial \\
latex & Lamport, www.ams.jhu.edu/~ers/learn-latex/ \\
svn & svn-book.pdf \\
BibDesk & \\
AIM@SHAPE & \\
LAPACK & Anderson; www.netlib.org/lapack/ \\
doxygen & doxygen\_manual-1.4.6.pdf \\ \hline
\end{tabular}

\end{document}






***************************
REGISTRATION: given two meshes, move them into correspondence;
given two point clouds, move them into correspondence
***************************

tech report on the ctr08 format

how to bridge a group of polygons, guided by a parent polygon
    (see MiscPolygon.cpp for general case; ctr08ToMesh.cpp for 2-1 case)

how to model shape at a branch
    (intermediate contour approach in branch.tex, motivated by Cassinian oval)
    (contours TO fragments, including branches of any kind 
	      TO fragments with only m-1 or 1-m branches 
	      TO fragments with all branches eliminated-i.e., reduced to 1-1)

obstacle avoiding interpolation (motivated by intermediate curves)

slice an existing surface optimally (which slice normal should be chosen? 
      can Morse theory help?)

******************************
model Birmingham 
    (USGS data, build its topology by hand using ctr08 format)
******************************

Fiedler segment a mesh, by weakening the mesh with curvature scissoring, then finishing
	the cut with Fiedler:
        1) mesh to weakened mesh, using curvature (use Princeton curvature code);
        2) Fiedler cut of weakened mesh (code is ready)

SIAM in 1 month: shape at a branch, Fiedler weakening
SGP in 2 months: shape at a branch, Fiedler weakening
CCCG in 3 months

resolved action item: build a Cassinian oval figure 8 merely from input of 2 circles

present action item (2/11/09): triangulate a branch: 3-1 case

David: simple code for project that focusses on code, not reading;
      - mesh segmentation, reconstruction, and registration are too mathematical
        and require too much reading
      - *a joint code-based image-analysis-based forensics project 
        with Gary, Alan and Tracy may be best: 
        have a meeting with Gary and Tracy; 
        not based on Fiedler analysis (problem is weakness in the Fiedler literature);
        alternative: segmentation and fitting of spam images, 
                      followed by analysis of similarity?
                      to analyse spam, technique would need to be automatic, not manual
      - *fitting code project (independent of spam?)
        e.g., read an image, sample a segmentation, fit a curve interactively as the
        segmentation samples arrive; *allow Hermite tangents 
        and derivative discontinuities (click with 1 or 2 depressed indicates drop in
        continuity by 1 or 2) to be specified too; 
        basically an aid to manual segmentation, indicating
        how the curve will look (rather than just the data points);
        *can we make it interactive enough so that as mouse moves the interpolating
        curve is shown (before the mouse click committing to the sample)?
        research issue: incremental update of the B-spline linear system with the
        addition of a new point so that curve may be updated efficiently (even though
        it depends on the entire pointset);
        goal is a representation of the image boundary; comparison of two spam images 
        based on these curve representations;
        test data = spam images; architectural footprints; city maps;
        thesis = open source and UML of (incrementally optimized) fitting code???
                 spam image boundaries?
        *given a curve, reduce to the minimal set of segments and control points that
        represent this curve (within some error), 
        so that two curves can be compared most easily and uniquely
      - (work on introductory programming with Java and Python using
        matrices and/or Bezier curves and/or projective geometry (HZ);
        but at a freshman level and developing clean open-source)
 

Lin: work on Munden project if it is funded; work to get a Google grant

Sagar: open source of Bezier code, leading to classical cubic B-spline interpolation,
       then Hermite interpolation, then contour code

Come along for a trip along the shape fantastic.
How about a glimpse of how to build an interpolating curve?
Consider a sequence of $n$ points.

website research areas:
shape modeling

research projects:
reconstruction of meshes and smooth surfaces from point data; 
reconstruction of meshes and smooth surfaces from curve data; 
software suites for shape modeling;
anatomical modeling;
registration of multiple shape models;
tangent space analysis of smooth surfaces, emphasizing dual representations
  and the calculation of bitangency;
smooth hulls and kernels;
visibility analysis of smooth scenes;
data structures for meshes;
mesh semantics, emphasizing mesh segmentation;
% intersection algorithms;
% motion planning; bisectors; algebraic curves;
Bezier and Hermite curves and surfaces;
semantic mining of online image databases, emphasizing canonical views;
multiple view geometry.

CIS brochure:

... perform research in shape modeling and computer graphics.
Of particular interest is the reconstruction of meshes and smooth surfaces,
anatomical modeling, mesh semantics, semantic mining of image databases, 
and software suites for geometry.

